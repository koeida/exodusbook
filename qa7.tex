\section{Quaestiō Augustīnī}

\marginpar{assiduē: continenter, non intermisse, saepissime}Assiduē Deus dīcit: ``Indūrābō cor Pharaōnis'': et velut causam īnfert cūr hoc
faciat: ``Indūrābō,'' inquit, ``cor Pharaōnis, et implēbō signa mea et portenta
mea in Aegyptō;'' tamquam necessāria fuerit obdūrātiō cordis Pharaōnis, ut
signa Deī multiplicārentur vel implērentur in Aegyptō. Ūtitur ergō Deus
\mimg{aug.jpg}{Augustinus}
bene cordibus malīs ad id quod vult ostendere bonīs vel quod factūrus est
bonīs. 

Et \marginpar{quamvis...qualitās}quamvīs ūnīuscuiusque cordis in
\marginpar{malitia, -ae: malignitas, mala natura}malitiā quālitās, id est, quāle
\marginpar{inolescere: crescere, in vel cum aliqua re adolescere,
coalescere}cor habeat ad malum, suō fīat vitiō, quod inolēvit ex arbitriō voluntātis;
ea tamen quālitāte malā ut hūc vel illūc moveātur, cum sīve hūc sīve illūc
male moveātur, causīs fit quibus animus prōpellitur: quae causae ut
existant vel nōn existant, nōn est in hominis potestāte; sed veniunt ex
occultā prōvidentiā, iūstissimā plānē et sapientissimā, ūniversum quod
creāvit dispōnentis et administrantis Deī. 

Ut ergō tāle cor habēret Pharaō,
quod patientiā Deī nōn movērētur ad pietātem, sed potius ad impietātem,
vitiī propriī fuit: quod vērō ea facta sunt, quibus cor suō vitiō tam
malignum resisteret iussiōnibus Deī (hoc est enim quod dīcitur indūrātum,
quia nōn flexibiliter cōnsentiēbat, sed īnflexibiliter resistēbat),
dispēnsātiōnis fuit dīvīnae, quā tālī cordī nōn sōlum nōn iniūsta, sed
ēvidenter iūsta poena parābātur, quā timentēs Deum corrigerentur.

Prōpositō
\marginpar{quippe (adv): coniunctio confirmans, qua rationem reddimus eius
quod praecessit}\marginpar{lucrum, -ī (n): pecunia tradita pro re peracta}quippe lucrō,
verbī grātiā, \linebreak
propter quod homicīdium committātur, aliter
\marginpar{facinus, facinōris (n): factum malum}avārus, aliter pecūniae contemptor movētur; ille scīlicet ad facinus
\marginpar{prōpositiō lucrī in potestāte alicuius illōrum nōn fuit}perpetrandum, ille ad cavendum: ipsīus tamen lucrī prōpositiō in alicuius
illōrum nōn fuit potestāte. 

Ita causae veniunt hominibus malīs, quae nōn
sunt quidem in eōrum potestāte, sed hoc dē illīs faciunt, quālēs eōs
invēnerint iam factōs propriīs vitiīs ex praeteritā voluntāte. Videndum
sānē est, utrum etiam sīc accipī possit: ``Ego indūrābō,'' tamquam dīceret:
Quam dūrum sit dēmōnstrābō. 
