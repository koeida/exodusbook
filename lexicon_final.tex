\lexentry{abigere:}{pellere}
\lexentry{abōminātiō, -ōnis (f):}{aliquid pessimum et malum}
\lexentry{abscondere:}{rem in aliquo loco ponere ut celetur}
\lexentry{absque =}{sine}
\lexentry{abyssus, -ī (f):}{aqua cuius fundus longissime a nobis est}
\lexentry{accingere:}{cingere}
\lexentry{acquiescere:}{fidem habere}
\lexentry{adaquāre:}{aquam dāre}
\lexentry{adhortārī:}{hortārī}
\lexentry{ad invicem:}{inter sē}
\lexentry{adiūrāre:}{iūrāre}
\lexentry{adiūtor, -ōris (m):}{qui adiuvat}
\lexentry{Adonaī:}{vocabulum Hebraicum significans `dominī'}
\lexentry{adoptāre:}{qui adoptat aliquem facit eum filium vel filiam suam}
\lexentry{adultus/a/um:}{iam vir/femina, non puer/puella}
\lexentry{advena, -ae (m/f):}{quī non civis est, quī in terrā aliā natus est}
\lexentry{adversārius/a/um:}{qui est ante seu contra nos positus}
\lexentry{aedēs, aedium (f pl.):}{aedificium}
\lexentry{Ægyptiacās =}{Ægyptiās}
\lexentry{Ægyptus, -ī (f)}{}
\lexentry{aeternus/a/um:}{quod neque principium neque finem temporis habet}
\lexentry{afflīctiō, -ōnis (f):}{rēs quae nocet vel valde non placet}
\lexentry{afflīgere:}{perturbāre, perdere}
\lexentry{agrestis, -e (adj) <}{ager}
\lexentry{aliēnigena, -ae (m/f):}{aliō locō natus}
\lexentry{alveus, -ī (m):}{fossa per quam fluvius fluit}
\lexentry{amāritūdo, -inis (f)}{\\< amārus/a/um $\leftrightarrow$ dulcis, -e}
\lexentry{amplius =}{plus}
\lexentry{angelus, -ī (m):}{res viva, sapiens, et potēns ā Deō missa quae agit quidquid Deus vult}
\lexentry{angustia, -ae (f):}{locus nōn latus}
\lexentry{animans, animantis (adj):}{vivēns}
\lexentry{anniculus/a/um:}{unius annī}
\lexentry{annūntiāre:}{nuntiāre}
\lexentry{antecēdere:}{ante īre}
\lexentry{aperīre vulvam:}{parī}
\lexentry{appēndō, appendere, appendi, appensum:}{pendere}
\lexentry{appropiāre:}{adīre}
\lexentry{arcānum, -ī (n):}{rēs quam paucī sciunt; rēs occultātur}
\lexentry{ārdēre:}{habēre ignem in sē}
\lexentry{āridus/a/um:}{sine aquā}
\lexentry{armentum, -ī (n):}{equus, bos, vel simile}
\lexentry{arripere:}{vī prehendere}
\lexentry{ascēnsor, -ōris (m):}{qui ascendit}
\lexentry{assistere:}{stāre ad aliquid vel prope aliquid}
\lexentry{assūmere:}{sumere ad sē; iungere sibi}
\lexentry{assus/a/um}{sine aquā vel aliā materiā fluentī coctus}
\lexentry{avāritia, -ae (f):}{amor pecuniae}
\lexentry{avus, -ī (m):}{pater patris vel matris}
\lexentry{azȳmus/a/um:}{sine materiā quae facit panem turgidum fiērī}
\lexentry{benedīcere:}{sanctum facere}
\lexentry{benedictiō, -ōnis (f) <}{benedicere: sanctum facere}
\lexentry{benedictus/a/um:}{laudatus}
\lexentry{bitūmen, -inis (n):}{māteria mollis atque ātra in terrā inventa quae pōnitur in rēbus nē aqua īnfluere possit}
\lexentry{brāchium}{ = bracchium}
\lexentry{cacūmen, cacūminis (m):}{altissima pars}
\lexentry{cæremōnia, -ae (f):}{res agendae ut sacra recte fiant}
\lexentry{calceāmentum, -ī (n):}{quod pedī induitur}
\lexentry{cālīgō, cālīginis (f):}{aer per quem difficile est vidēre}
\lexentry{callidus/a/um:}{doctus, praecipue rebus agendis, non rebus ex librīs}
\lexentry{canālis, -is (m/f):}{rēs in quam aqua funditur ut animal bibat}
\lexentry{captīva, -ae (f):}{qui captus, qui vinctus est}
\lexentry{cārectum, -ī (n):}{locus caricibus plenus}
\lexentry{cārex, cāricis (f):}{herba acuta et durissima}
\lexentry{castramētārī:}{castra ponere}
\lexentry{celebrāre:}{agere quod oportet agere die Deō constitutō}
\lexentry{centuriō, -ōnis (m):}{dux centum hominum}
\lexentry{cervīx, cervīcis (f):}{posterior pars collī; collum}
\lexentry{chorus, -ī (m):}{actus corporis movendi propter hominem cantantem vel similem}
\lexentry{cinis, cineris (m):}{postquam ignis lignum totum consumpsit, cinis manet}
\lexentry{circuitus, -ūs (m):}{locus circum alterum locum}
\lexentry{circumcīdere:}{removēre secandō partem pēnis}
\lexentry{clangere:}{facere sonum magnum ut tuba}
\lexentry{clangor, -ōris (m):}{strepitus}
\lexentry{coarctāre:}{in angustum cogere}
\lexentry{cœtus, -ūs (m):}{grex hominum}
\lexentry{cognātus/a/um (adj):}{qui pars eiusdem familiae est}
\lexentry{colligere:}{invenīre rēs quae hīc illīc sunt, deinde sumere eās atque in unō locō ponere}
\lexentry{colōnia, -ae (f):}{locus in quō hominēs domūs novās confēcērunt et agrōs colere incēpērunt}
\lexentry{combūrere:}{igne perdere}
\lexentry{comburō, comburere, combussī, combussum:}{accendere}
\lexentry{comedere:}{totum ēsse; ēsse}
\lexentry{commodāre:}{libenter cum aliquō facere, libenter rēs dāre}
\lexentry{compellere:}{vī facere ut aliquis aliquid agat}
\lexentry{comprehendere:}{prehendere}
\lexentry{computrēscere:}{facere malum odorem (id quod nasus sentit) propter rem mortuam}
\lexentry{concipere:}{femina quae concipit efficit ut infans in eā incipit crescere}
\lexentry{conclūdere:}{claudere}
\lexentry{concupiscere:}{valdē cupere}
\lexentry{concussus/a/um:}{pulsātus; turbātus}
\lexentry{condere:}{constituere, facere}
\lexentry{condīcere:}{aliquid inter aliquōs constituere; promittere}
\lexentry{cōnflātilis, -e (adj):}{quod flandō factum est}
\lexentry{cōnfodere:}{vulnerāre}
\lexentry{cōnfortāre:}{fortem facere, consolārī}
\lexentry{cōnfringere:}{ūnā frangere}
\lexentry{cōnfugere:}{fugere}
\lexentry{congregāre:}{in unum locum cōgere vel ferre vel ducere}
\lexentry{congregātiō, -ōnis (f):}{grex hominum}
\lexentry{cōnsecrāre:}{sacrum facere}
\lexentry{cōnservāre:}{servāre}
\lexentry{cōnsīderāns ēventum reī:}{exspectans ut videat quid futurum sit}
\lexentry{cōnspergō, cōnspergere, cōnspersī, conspersum:}{spargere}
\lexentry{cōnsurgere:}{surgere}
\lexentry{conterō, conterere, contrīvī, contrītum:}{perdere}
\lexentry{contestāre populum:}{fac tē testem verbōrum meōrum populō; narra populō ea quae tibi dixī}
\lexentry{cooperīre:}{operīre}
\lexentry{coriandrum, -ī (n):}{quaedam herba}
\lexentry{corrōdere:}{consumere}
\lexentry{corruptus/a/um:}{id quod in peius mutatur}
\lexentry{coturnīx cooperuit castra:}{coturnīcēs cooperuērunt castra}
\lexentry{crepīdo, -inis (f):}{ripa fluminis lapidibus strata}
\lexentry{crūdus/a/um:}{non coctus}
\lexentry{cultus, -ūs (m)}{cura Deī, ornamenta, vestimenta}
\lexentry{decānus, -ī (m):}{dux decem hominum}
\lexentry{dēferre:}{ferre aliquid aliquō}
\lexentry{dēicere:}{deorsum iacere}
\lexentry{densus/a/um:}{cuius partēs inter sē premunt}
\lexentry{dēprecārī:}{aliquem aut aliquid per preces ā periculō līberāre}
\lexentry{dēsertum, -ī (n):}{locus ubi paene nihil est; locus sine imbre et aquā}
\lexentry{dēsuper:}{ē locō altiore}
\lexentry{dīlūculum, -ī (n)}{prima lux diei}
\lexentry{disceptātiō, -ōnis (f):}{e.g, unus dicit alterum malum fecisse, et alter se defendit -- hoc est disceptātiō}
\lexentry{discurrere:}{hūc illūc currere}
\lexentry{disperdere:}{valde perdere}
\lexentry{dispergō, dispergere, dispersī, dispersus:}{in multās et variās partēs vel locōs spargere}
\lexentry{dīversōrium, -ī (n):}{aedificium in quō hominēs in itinere possunt dormīre}
\lexentry{dīversus/a/um:}{nōn similis}
\lexentry{dīvīsiō, -ōnis (f) <}{dīvidere}
\lexentry{dracō, -ōnis (m):}{coluber}
\lexentry{dūdum:}{nuper}
\lexentry{duplex, duplicis (adj):}{bis tantus}
\lexentry{duplus/a/um:}{bis tantus}
\lexentry{duritia, -ae (f) $<$}{durus/a/um}
\lexentry{ēbullīre:}{mittere bullās}
\lexentry{efferre:}{extra ferre, proferre}
\lexentry{ēgressiō, -ōnis (f) <}{ēgredī}
\lexentry{ēlegāns =}{pulcher}
\lexentry{ēlevāre:}{tollere; sursum levāre}
\lexentry{eloquēns:}{bene loquēns}
\lexentry{emptitius =}{emptus}
\lexentry{ēn:}{ecce}
\lexentry{ephī (n) (indec):}{mensura Hebraeōrum}
\lexentry{ergastulum, -ī:}{carcer rusticus}
\lexentry{ēvāgīnāre:}{ē vāgīnā educere}
\lexentry{exāctōr, -ī:}{quī curat ut opera perficiantur}
\lexentry{exaltāre:}{in altum levāre}
\lexentry{exaltātiō, -ōnis (f) < }{exaltāre (in altum levāre)}
\lexentry{excelsus/a/um:}{altus}
\lexentry{exceptīs hīs:}{praeter hās}
\lexentry{expectāre =}{exspectāre}
\lexentry{explēre:}{perficere}
\lexentry{extrēmus/a/um:}{ultimus, postremus}
\lexentry{fabricatus/a/um:}{confectus}
\lexentry{famula, -ae (f):}{ancilla}
\lexentry{famulātus, -ūs (m):}{servitūs}
\lexentry{famulus, -ī (m)}{servus}
\lexentry{fār, farris (n):}{genus frumentī}
\lexentry{farīna, -ae (f):}{materia, aquā impositā, ex quā panis factus est}
\lexentry{fasciculum, -ī (n):}{multum lignī vel materiae aliae vinctum}
\lexentry{fermentō, -āre, -āvī, -ātum:}{ponere fermentum in aliquō}
\lexentry{fermentum, -ī (n):}{materia quae efficit ut panis turgidus fiat}
\lexentry{festīnanter =}{celeriter}
\lexentry{festīnus/a/um:}{celeriter aliquid agens}
\lexentry{fēstīvitās, -ātis (f):}{dies laetus deō constitutus}
\lexentry{firmāre:}{munīre; fortem facere}
\lexentry{firmus/a/um:}{durus; quod diu manet et nōn mutātur}
\lexentry{flagellāre:}{flagellīs pulsāre}
\lexentry{foetēre:}{emittere malum odorem}
\lexentry{formāre:}{formam alicui reī dāre, facere}
\lexentry{formīdō, -ōnis (f):}{timor}
\lexentry{fortitudō, -inis (f) <}{fortis}
\lexentry{fūmāre:}{fumum mittere}
\lexentry{fūmāre:}{fūmum mittere}
\lexentry{fundāre:}{constituere}
\lexentry{furōr, -ōris (m):}{qui habet `furorem' in sē valde perturbatur}
\lexentry{fūrtum, -ī (n):}{e.g, si capis pecuniam vel aliquid alius, hoc est furtum}
\lexentry{fūsōrius/a/um:}{quod funditur}
\lexentry{gemitus, -ūs (m) <}{gemere: facere sonum dolendī}
\lexentry{generātio, -ōnis (f):}{e.g: una generatio parentibus constat, līberīs eōrum altera generatio constat, etc...}
\lexentry{germināre:}{emittere germina (ea quae ex herbīs veniunt: semen, fructus, cēt.)}
\lexentry{glōrificāre:}{gloriosum facere}
\lexentry{gomor:}{mensura Hebraeōrum}
\lexentry{gradī:}{gradum facere, ambulāre}
\lexentry{grandō, grandinis (f):}{aqua frigidissima lapidī similis quae cadit dē caelō}
\lexentry{gustus, -ūs (m) < gustāre}{}
\lexentry{habitābilis, -e (adj):}{habitārī potest}
\lexentry{habitāculum, -ī (n):}{locus in quō aliquis habitat}
\lexentry{habitātiō, -ōnis (f) <}{habitāre}
\lexentry{habitātor, -ōris (m):}{incola}
\lexentry{hāc vice =}{hōc tempore}
\lexentry{haerēditās, -ātis (f):}{e.g, pater meus pecuniosus moritur et pecunia eius mihi datur - haec pecunia haerēditās vocātur}
\lexentry{haurīre:}{sumere ex loco ubi aqua est}
\lexentry{holocaustum, -ī (n):}{sacrificium igne factum}
\lexentry{honōrāre:}{colere aliquem propter eius virtutēs vel rēs bene factās}
\lexentry{hordeum, -ī (n):}{genus frumentī}
\lexentry{horribilis, -e (adj):}{timendus/a/um}
\lexentry{hostia, -ae (f):}{animal quod in sacrificiīs interficitur}
\lexentry{huiuscemodī =}{huius modī}
\lexentry{humerōs =}{umerōs}
\lexentry{hyssōpum, -ī (n):}{herba quaedam}
\lexentry{iaculum, -ī (n):}{quiquid iacī potest ut aliquis vulnerētur}
\lexentry{idcircō:}{proptereā, ideō}
\lexentry{ignōminia, -ae (f):}{id quod minime decet}
\lexentry{illūdere:}{deridēre}
\lexentry{illūmināre:}{illustrāre}
\lexentry{immēnsus/a/um:}{tam magnum ut nemo possit pro certo invenire quam magnum sit}
\lexentry{immitere:}{intus mittere, inducere}
\lexentry{immōbilis, -e (adj):}{quī nōn movētur vel moverī non potest}
\lexentry{immolāre:}{sacrificium facere}
\lexentry{immūtātus/a/um:}{non mutatus}
\lexentry{impedītus/a/um:}{difficile ūtī}
\lexentry{impius/a/um:}{qui deum nōn colit}
\lexentry{in æternum:}{semper}
\lexentry{inaurēs, inaurium (f. pl.):}{ornamenta in auribus posita}
\lexentry{incalesco, incalescere, incaluī:}{valde calidus fiērī}
\lexentry{incantātiō, -ōnis (f) < incantāre:}{dicere verba quibus res mirabiles fiant}
\lexentry{in-circumcīsus/a/um:}{non circumcīsus}
\lexentry{indicāre:}{ostendere, monstrāre}
\lexentry{indigena, -ae (adj):}{natus in eō lōcō}
\lexentry{indignārī:}{aliquid graviter animō ferre; ferre non posse}
\lexentry{inducere:}{ducere in aliquem locum}
\lexentry{indūrāre:}{durum facere}
\lexentry{īnfantulus/a/um:}{parvus īnfans}
\lexentry{īnferre <}{in + ferre}
\lexentry{ingemīscere:}{dolēre}
\lexentry{ingravāre:}{facere gravem}
\lexentry{ingredī:}{intus īre}
\lexentry{ingruere:}{cum vi accedere}
\lexentry{in hāc vice:}{hōc tempore}
\lexentry{inīquitās, -ātis (f):}{rēs mala ab homine facta; mala anima propter rem malam factam}
\lexentry{innumerābilis, -e (adj):}{quī numerārī nōn potest}
\lexentry{in similitūdinem pruīnae:}{similis pruīnae}
\lexentry{īnsons, īnsontis (adj):}{quī rem malam nōn fēcit}
\lexentry{īnstāre:}{in vel super aliquā rē stāre}
\lexentry{īnstar + gen:}{similis}
\lexentry{īnsuper:}{praeterea}
\lexentry{interiōr, -ōris (adj.):}{magis intus}
\lexentry{intrōdūcere:}{ducere intro}
\lexentry{introīre:}{intus īre; ingredī}
\lexentry{in vānum:}{frūstrā}
\lexentry{invicem:}{Aegyptiī, deinde filiī Isrāel}
\lexentry{involvere:}{facere ut aliquid anguste circumdetur}
\lexentry{irruere:}{in aliquid vi se mittere}
\lexentry{iudicāre:}{quod iudex agit}
\lexentry{iūdicium, -ī (n):}{quod iudex tradit}
\lexentry{iūmentum, -ī (n):}{animal utile ad gerendum vel trahendum e.g, equī, bovēs, et cetera}
\lexentry{iurgārī:}{querī}
\lexentry{iūrgium, -ī (n) <}{iūrgārī}
\lexentry{labium:}{labrum}
\lexentry{lædere:}{vulnerāre}
\lexentry{lampas, lampadis (f):}{ignis}
\lexentry{lapidāre:}{lapidēs iacere}
\lexentry{lapideus/a/um:}{ex lapidibus factum}
\lexentry{lassāre:}{fatigāre}
\lexentry{laudābilis, -is (adj):}{qui laudārī potest}
\lexentry{lectulus, -ī (m):}{lectus}
\lexentry{lēgitimus/a/um:}{qui secundum legēs est; iustus; verus}
\lexentry{leprōsus/a/um:}{homo leprōsus aegrotat -- cutis (id quod operit totum corpus) aegra est}
\lexentry{lignum:}{arbor}
\lexentry{līnere:}{imponere rem mollem premendō ac tergendō}
\lexentry{linum, -ī (n):}{herba utilis ad vestimenta facienda}
\lexentry{liquefiērī:}{fiērī res fluens sicut aqua}
\lexentry{littus, -ōris (n):}{terra apud mare quae ā fluctibus pulsatur}
\lexentry{loca =}{locī (nom. pl.)}
\lexentry{longævus/a/um:}{senex}
\lexentry{lūgeō, lugēre, lūxī, luctum}{}
\lexentry{lutum, -ī (n):}{terra humida}
\lexentry{mactāre:}{interficere, occidere}
\lexentry{macula, -ae (f):}{pars in quā color proprius abest}
\lexentry{magnitūdō, -ūdinis (f):}{quam magnum aliquid est}
\lexentry{maleficus, -ī (m):}{homo quī rēs mirabilēs facere potest}
\lexentry{mandāre:}{tradere}
\lexentry{mandātum, -ī (n)}{id quod imperatum vel traditum est}
\lexentry{mandūcāre:}{cibum dentibus iterum iterumque premere}
\lexentry{mānsiō, -ōnis (f):}{locus ubi iter facientes noctu manent}
\lexentry{masculus/a/um:}{masculīnus}
\lexentry{mas, maris (m):}{homo (vel animal) masculinus}
\lexentry{mātūtīnus/a/um:}{quod in mane vel circum mane est}
\lexentry{mementō:}{memoriā tene}
\lexentry{mementōte =}{memoriā tenēte}
\lexentry{memoriāle, -is (n):}{id quod memorandum est}
\lexentry{mendāx, -ācis (adj):}{mentiēns}
\lexentry{mēnsūra, -ae (f):}{e.g, ``da mihi decem vasa aquae plena!'' Vas dicitur mēnsūra aquae}
\lexentry{mercēnārius, -ī (m):}{cui merces datur ut opus faciat}
\lexentry{mētior, mētīrī, mēnsus sum:}{e.g, homo habet magnum vas aquae plenum et oportet eum fundere aquam in decem alia parva vasa. Dicimus hominem mētīrī aquam}
\lexentry{micāre:}{lucem mittere ut ignis, stellae, et cetera}
\lexentry{mināre:}{cōgere animālia}
\lexentry{minūtus/a/um:}{parvus}
\lexentry{miserērī:}{dolorem sentīre propter miserum alterum}
\lexentry{misericordia, -ae (f):}{quod sentimus dum vidēmus aliquem patī}
\lexentry{miserta (+ gen.):}{tristis propter aliquem miserum}
\lexentry{mīsit Pharaō (aliquem) ad videndum}{}
\lexentry{mœchārī:}{iacēre in lectō cum uxore vel marītō alius}
\lexentry{mola, -ae (f):}{instrumentum grave quō frumentum premitur in minimās partēs}
\lexentry{monimentum, -ī (n):}{quod nos monet}
\lexentry{monumentum, -ī (n):}{quod nōs monet}
\lexentry{multiplicāre:}{multum facere; augēre}
\lexentry{murmurāre:}{loqui de re quae non placet}
\lexentry{murmurātio, -ōnis (f) <}{murmurāre}
\lexentry{murmur, murmuris (n):}{sonus hominis qui loquitur de re quae non placet}
\lexentry{mussitāre:}{parvā voce dicere}
\lexentry{mūtīre:}{parvā voce sonum facere vel loquī}
\lexentry{mūtuō salūtāvērunt:}{Hic illum salūtāvit et ille hunc salūtāvit}
\lexentry{nebula, -ae (f):}{nūbēs prope terram}
\lexentry{necdum:}{nondum}
\lexentry{ne forte...multiplicētur\\...addātur...egrediātur}{}
\lexentry{nēquāquam:}{minimē, nullō modō}
\lexentry{nēquitia, -ae (f):}{e.g, homo saepe mentitur: hoc est nēquitia eius. Homo saepe iratus est: hoc est nēquitia eius}
\lexentry{nōn fēcērunt iuxtā praeceptum regis:}{nōn pāruērunt praeceptō rēgis}
\lexentry{noxa, -ae (f):}{quod nocet}
\lexentry{nūdiustertius:}{ante duōs diēs}
\lexentry{nūtrīre:}{dāre lac}
\lexentry{obēdiēns, -entis (adj):}{parēns}
\lexentry{obrigescō, obrigescere, obriguī:}{durus fierī}
\lexentry{obruere:}{operīre}
\lexentry{obsecrāre:}{orāre, precārī}
\lexentry{observābilis, -e (adj):}{quod potest observārī}
\lexentry{observāre:}{colere, servāre}
\lexentry{obstetrix, -īcis (f):}{femina quae iuvat feminam gravidam parere}
\lexentry{obtinēre:}{tenēre, habēre, possidēre}
\lexentry{occursus, -ūs (m) <}{occurere}
\lexentry{odōr, odōris (m):}{quod nasus sentit}
\lexentry{omnīnō:}{vere, certe}
\lexentry{omnipotens, -entis (adj):}{habens omnem potestatem}
\lexentry{omnipotēns, omnipotentis (adj):}{habens omnem potestatem}
\lexentry{onus, oneris (n):}{res quam difficile est portare vel agere. e.g: saccus lapidibus plenus, labor difficilis}
\lexentry{opprimere:}{premere aliquem ne surgere possit}
\lexentry{pācificum (sacrificium):}{sacrificium in quō parva pars sacrificii Deō datur et magna pars populō}
\lexentry{pācificus/a/um:}{pacem faciens}
\lexentry{pactum, -ī (n):}{quod inter aliquōs convenit}
\lexentry{palam}{$\leftrightarrow$ occulte}
\lexentry{palpārī:}{leviter tangere}
\lexentry{palūs, palūdis (f):}{aqua nōn fluēns}
\lexentry{pangō, pangere, pepigisse, pactum:}{constituere, statuere}
\lexentry{pāpiliō, -ōnis (m):}{tabernaculum quod formam animalis eodem nomine habet}
\lexentry{papyrio, -ōnis (m):}{locus multīs cum papyrīs}
\lexentry{pariter:}{aequē}
\lexentry{partior, partīrī, partitus:}{partēs facere, dīvidere}
\lexentry{partus, -ūs (m) <}{parere}
\lexentry{paulātim:}{tardē plus et plus et plus vel magis et magis et magis}
\lexentry{paululum =}{paulum}
\lexentry{pavor, -ōris (m):}{timor}
\lexentry{peccāre:}{contra legem agere}
\lexentry{peccātum, -ī:}{res contra legem acta}
\lexentry{pēnūria, -ae (f):}{id quod paene minus sit quam necesse est}
\lexentry{percussōr, -ōris (m):}{quī percutit}
\lexentry{percutere:}{pulsare, afficere, et quasi vulnerare dolore}
\lexentry{per-dūcere:}{trahere}
\lexentry{peregrinārī:}{iter facere per aliēnōs locōs, patria procul abīre}
\lexentry{peregrīnātiō, -ōnis (f)}{ < peregrinārī: iter facere per aliēna loca, patria procul abīre}
\lexentry{peregrīnus/a/um:}{quī ex aliā terrā vēnit}
\lexentry{perstrepere:}{strepitum magnum facere}
\lexentry{pertinēre:}{e.g, colloquium ad mē pertinet sī hominēs loquuntur dē mē}
\lexentry{per-transīre:}{transīre per aliquid, praeter aliquid īre}
\lexentry{pestis, -is (f):}{aliquid corporī malum: e.g, aeger fiērī}
\lexentry{petra, -ae (f):}{lapis}
\lexentry{Phase:}{vocabulum Hebraicum ``trānsitus'' significans}
\lexentry{pilum, -ī (n):}{baculum minimum quō homo herbās aliāsque rēs frangit in parvō poculō}
\lexentry{pix, picis (f):}{māteria bitūminī similis mollis atque ātra ex parte arboris facta}
\lexentry{plācābilis, -e (adj):}{qui facile tranquillus fiērī potest}
\lexentry{plāga, -ae (f):}{e.g, unus pulsat alterum: eo tempore cum pugnus corpus tangit `plāga' vocatur}
\lexentry{plantāre:}{ponere herbam in terrā ut crescat}
\lexentry{plēbs, plēbis (f):}{populus; praecipue hominēs quī pecuniosī non sunt ac minimam potestatem habent}
\lexentry{pluere:}{imbrem facere}
\lexentry{plumbum, -ī (n):}{genus metallī}
\lexentry{pluvia, -ae (f):}{imber}
\lexentry{pœnitet:}{e.g, si aliquem nocui et nunc me non nocuisse volo, me poenitet nocendī.}
\lexentry{polluere:}{foedum facere}
\lexentry{possessiō, -ōnis (f):}{id quod possidētur}
\lexentry{potēns, potentis (adj):}{qui multas res agere potest: e.g, dux, rēx}
\lexentry{pōtus, -ūs (m):}{pōtiō}
\lexentry{præbeō, praebēre, praebuī, praebitum:}{offerre}
\lexentry{præcēdere:}{ante īre}
\lexentry{præceptum, -ī:}{quod praecipitur}
\lexentry{præcinere:}{ante cantāre}
\lexentry{præcipere:}{constituere, imperare}
\lexentry{præcō, præcōnis (m):}{quī magnā voce aliquid nuntiat}
\lexentry{præcursor, -ōris (m)}{qui currit prae aliīs}
\lexentry{præ-esse:}{esse prae alios; dux esse}
\lexentry{præfectus, -ī (m):}{homo qui alicuī reī vel negotiō praepositus est}
\lexentry{præparāre:}{parāre, ante parāre}
\lexentry{præpūtium, -ī (n):}{illa pars pēnis quae circumcīditur}
\lexentry{præsertim:}{praecipue}
\lexentry{præsertim:}{praecipue}
\lexentry{præstōlārī:}{stāre ante aliquem et expectāre}
\lexentry{prīmitiae, -ārum (f):}{primae rēs quae ex agrīs metuntur}
\lexentry{prīmitīvus/a/um:}{primo adveniens}
\lexentry{prīmōgenitus/a/um:}{quī vel quae prīmus natus est}
\lexentry{prīnceps, prīncipis (m):}{rex; dux}
\lexentry{profundum, -ī (n):}{fundus}
\lexentry{profundus, -ī (m):}{fundus}
\lexentry{prōicere:}{ante aut procul iacere}
\lexentry{prōlixus/a/um:}{longus}
\lexentry{prōmiscuus/a/um:}{mixtum}
\lexentry{prōnus/a/um:}{quod flectitur ad aliquid ante sē}
\lexentry{prophētissa, -ae (f):}{femina quae bene deum intellegit et de deō loquitur}
\lexentry{prōtegere:}{defendere}
\lexentry{prōvidēre:}{curāre ut aliquid habeās}
\lexentry{pruīna, -ae (f):}{rōs frigidissimus}
\lexentry{pulmentum, -ī (n):}{cibus}
\lexentry{pulvis, pulveris (m):}{terra minima et non ūmida}
\lexentry{quæsō:}{homo dicit `quæsō' cum velit verba sua minus duriora viderī}
\lexentry{quaestiō, -ōnis (f):}{quod interrogatur, praecipuē cum respondēre difficile est: e.g, ``Estne Deus?'', ``Quid est hominibus bonum?'', etc.}
\lexentry{quasi germinantēs:}{i.e, non vero germinantēs (quia hominēs non germinant) sed vidērī aliquō modō germināre}
\lexentry{queant =}{possint}
\lexentry{queō, quīre, quīvī, quitum:}{posse}
\lexentry{quīnam =}{quī}
\lexentry{quīnquāgēnāriō, -ōnis (m):}{dux quīnquāgīnta hominum}
\lexentry{quotīdiē =}{cotīdiē}
\lexentry{rādīx montis $\leftrightarrow$}{altissima pars montis}
\lexentry{reātus, -ūs (m):}{peccatum propter quod homo puniendus est}
\lexentry{recordātiō, -ōnis (f) <}{recordārī: cogitāre dē aliquā memoriā}
\lexentry{recordor, -ārī, -ātus sum:}{meminisse}
\lexentry{rēgum, rēgī (n):}{quod rēx rēgit}
\lexentry{religiō, -ōnis (f):}{rītus (res agendae ut sacra recte fiant)}
\lexentry{reliquiae, -ārum (f):}{pauca illa quae ex aliquā rē relicta sunt}
\lexentry{rēnēs accingere:}{induere vestem quae cingit corpus circiter rēnēs}
\lexentry{renuere:}{capite movendō significāre sē nolle}
\lexentry{reputāre:}{cogitāre}
\lexentry{requiēs, requiēī (f):}{tempus quō homo non laborat}
\lexentry{reservāre:}{servāre, conservāre}
\lexentry{residuus/a/um:}{quod reliquum est}
\lexentry{respergere:}{aspergere}
\lexentry{respicere:}{aspicere iuvandī causā}
\lexentry{retrahere:}{trahere ab aliquō}
\lexentry{revēlāre:}{efficere ut aliquid occultum appāreat}
\lexentry{rīpa, -ae (f):}{terra tangēns flumen}
\lexentry{rītus, -ūs (m):}{res agendae ut sacra recte fiant}
\lexentry{rixāre:}{certāre}
\lexentry{rōborāre:}{virēs dāre}
\lexentry{rōbustus/a/um:}{durus, valēns}
\lexentry{rūrsum =}{rūrsus}
\lexentry{sabbatīzāre:}{sabbatum colere}
\lexentry{sabbatum, -ī (n):}{dies septimus quō non licet hominibus laborāre}
\lexentry{sacerdōs, -ōtis (m):}{quī res sacrās agit vel qui ducit aliōs in rēbus sacrīs agendīs}
\lexentry{sacerdōtālis, -e (adj) <}{sacerdōs}
\lexentry{sacrificāre:}{animal interficere vel rem offerre ut placeat deo}
\lexentry{sānctificāre:}{sanctum facere}
\lexentry{sānctitās, -ātis (f) $<$}{sanctus}
\lexentry{sānctuārium, -ī (n):}{sacer locus}
\lexentry{sānctus/a/um:}{deō pertinens}
\lexentry{sapienter (adv) <}{sapiēns}
\lexentry{sapphīrinus/a/um <}{sapphīrus}
\lexentry{saturāre:}{tam multum ēsse ut fortasse dicas ``bene, bene, satis est!''}
\lexentry{saturitās, -ātis (f)}{< satis}
\lexentry{saxum, -ī (n):}{lapis}
\lexentry{scandalum, -ī (n):}{aliquid in terrā positum ut homo id pede pulset et cadat}
\lexentry{scatēre:}{plēnum esse}
\lexentry{scientia, -ae (f):}{id quod scītur}
\lexentry{sciniphēs, -um (f. pl.)}{minima et molesta animalia volantia}
\lexentry{scirpeus/a/um:}{ex scirpīs factus}
\lexentry{scrīptūra, -ae (f):}{verba scrīpta}
\lexentry{sculptile, -is (n):}{rēs secandō facta ut videatur alicuī reī similis, e.g: signum}
\lexentry{sēmetipsum =}{sē}
\lexentry{sempiternus/a/um:}{perpetuus}
\lexentry{seniōr, -ōris (adj):}{magis senex; hīc seniōrēs virī, id est, principēs}
\lexentry{sepulchrum, -ī (n):}{locus ubi mortuus positus est}
\lexentry{sērōtinus/a/um:}{quod tarde maturum fit}
\lexentry{sexus, -ūs (m):}{id quō animal masculinum vel femininum esse dicitur}
\lexentry{sīcubi:}{sī in aliquō locō}
\lexentry{similiter (adv)}{$<$ similis}
\lexentry{similitūdō, -inis (f):}{rēs quae formam aliae reī similem habet; imāgō}
\lexentry{socer, socerī:}{pater uxōris vel pater marītī}
\lexentry{sōlemnis, -e (adj):}{sōlemnis dicitur de rēbus deō constitutibus quae certīs temporibus quotannis fit}
\lexentry{sōlemnitās, -ātis (f):}{diēs quī certīs temporibus quotannis fit quō homines deum serviunt}
\lexentry{sōlitūdo, -inis ($<$ solus/a/um):}{deserta; locus ubi nemo vel unus habitat}
\lexentry{sollicitāre:}{allicere}
\lexentry{sonitus, -ūs (m):}{sonus magnus, strepitus}
\lexentry{spatiōsus/a/um:}{magnus}
\lexentry{spīritus, -ūs (m):}{animus}
\lexentry{spolia, -ae (f):}{quod de hoste victo detrahitur}
\lexentry{spoliāre:}{capere rēs aliōrum hominum}
\lexentry{spōnsiō, -ōnis (f):}{id quod promissum est}
\lexentry{spōnsus/a:}{quī mox maritus vel uxor erit}
\lexentry{stillāre:}{minima pars aquae mittere}
\lexentry{stipula, -ae (f):}{media ac longa pars herbae ex quā aliae partēs extendunt}
\lexentry{stirps, stirpis (m/f):}{e.g, frater matris est `stirpis meae', filia sororis patris quoque est `stirpis meae'. Quisquis ex familiā meā est, `stirpis meae' vocatur}
\lexentry{strātus, -ūs (m)}{vestis super lectum sternitur}
\lexentry{strēnuus/a/um:}{impiger, industrius}
\lexentry{subcinerīcius pānis:}{panis sub cinere coctus}
\lexentry{subicere (sub + iacere):}{facere ut aliquis alicui pareat}
\lexentry{subter $\leftrightarrow$}{super}
\lexentry{sub-vertere:}{vertere aliquid ut summa pars ad fundum vertatur et fundus ad summam partem vertatur}
\lexentry{succrēscere (sub + crēscere):}{ab īmō crēscere}
\lexentry{sufficere:}{satis esse}
\lexentry{suggerere:}{offerre consilium}
\lexentry{superāre:}{vincere}
\lexentry{superficiēs, -eī (f):}{summa pars reī quae operit aliās īnferiorēs partēs}
\lexentry{supervēnīre:}{subitō adesse nocendī causā}
\lexentry{suscipere:}{sumere, capere}
\lexentry{sustentāre:}{sustinēre}
\lexentry{tantummodo =}{solum}
\lexentry{tenebrōsus/a/um:}{obscurus}
\lexentry{tentāre:}{conārī aliquid ut videatur quid fiat}
\lexentry{tentātiō, -ōnis (f) <}{tentāre}
\lexentry{tentōrium, -ī (n):}{tabernaculum}
\lexentry{terminus, -ī (m):}{finis terrae}
\lexentry{terror, -ōris (m):}{magnus timor}
\lexentry{testificārī:}{testis esse}
\lexentry{testimōnium, -ī (n):}{id quod ā teste dīcitur}
\lexentry{tingere:}{humidum facere}
\lexentry{titulus, -ī (m):}{columna}
\lexentry{tonitruum, -ī (n):}{tonitrus}
\lexentry{trānscendere:}{transīre}
\lexentry{tribuere:}{dāre}
\lexentry{tribūnus, -ī (m):}{dux multōrum}
\lexentry{trīticum, -ī (n):}{genus frumentī}
\lexentry{turgēre:}{esse turgidum}
\lexentry{turma, -ae (f):}{multitūdō}
\lexentry{turpitūdō, -inis (f) <}{turpis}
\lexentry{tūsus/a/um:}{pulsatus}
\lexentry{ubicumque:}{in omne locō in quō...}
\lexentry{ulciscī:}{e.g, Agricola pastorem necat, deinde amicus pastoris agricolam necat, id est, amicus pastoris `ulciscitur' agricolam.}
\lexentry{ulcus, ulceris (n):}{vulnus turgidum in homine plenum reī foedae}
\lexentry{ultiō, -ōnis (f) <}{ulciscī}
\lexentry{unda, -ae (f):}{parvus fluctus}
\lexentry{ūnusquisque:}{singulī hominēs}
\lexentry{urbēs tabernāculōrum:}{ubi rēs pretiosae regis et aliae rēs positae sunt}
\lexentry{ūrere:}{id quod ignis agit}
\lexentry{urgēre:}{valde hortārī; cogere}
\lexentry{usquequō:}{usque ad quod tempus?}
\lexentry{vacāre:}{sine negotiīs esse}
\lexentry{vacuuus/a/um:}{sine rēbus (cibō, vestimentīs, pecuniā)}
\lexentry{vādere:}{īre, ambulāre}
\lexentry{vastāre:}{percutere}
\lexentry{vehemēns, -entis (adj):}{ferox, iratus, severus}
\lexentry{vehemens:}{ferox, iratus, severus}
\lexentry{vēlox, -ōcis:}{celer}
\lexentry{venerābilis, -e (adj):}{dignum colī}
\lexentry{vertex, verticis (m):}{altissima pars}
\lexentry{vērumtamen:}{sed tamen}
\lexentry{vēscī:}{cibō utī}
\lexentry{vēscī:}{ēsse}
\lexentry{vēsīca, ae (f):}{genus ulceris}
\lexentry{vespera, -ae (f):}{vesper}
\lexentry{vīcīnus/a/um:}{quī vel quae prope habitat}
\lexentry{victima, -ae (f):}{animal sacrificiō statutum}
\lexentry{vīgēsimam =}{vīcēsimam}
\lexentry{vīsiō, -ōnis (f) <}{vidēre}
\lexentry{vīsitāre:}{vīsum īre}
\lexentry{vitulus, -ī (m):}{bōs parvus}
\lexentry{vōciferārī:}{valde exclamāre}
\lexentry{vōciferātiō, -ōnis (f):}{clamor}
\lexentry{volūmen, volūminis (n):}{liber}
\lexentry{vulgus, -ī (n):}{multitudo, turba}
\lexentry{vulva, -ae (f):}{pars corporis feminae ubi infans crescit antequam paritur}
\lexentry{zēlōtēs, -ae (m):}{qui amat aliquem et non vult eum aliquem amāre}
