\chapter{}

\titleimg{frogtitle}

\mktitle{Capitulum Octavum}
\thispagestyle{empty}

  Dīxit quoque Dominus ad Moysēn: \mpp{ingredī:}{intus ire}``Ingredere 
ad \mimg{frog.png}{rāna, -ae (f)}Pharaōnem, et dīcēs ad eum: Hæc dīcit
Dominus: Dīmitte populum meum, ut \mpp{sacrificāre:}{animal interficere vel rem offerre ut placeat deo}sacrificet mihi: sīn autem nōluerīs
dīmittere, ecce ego percutiam omnēs \mpp{terminus, -ī:}{finis terrae}terminōs tuōs
rānīs,  et \mpp{ēbullīre:}{mittere bullās}\mimg{bubble.png}{bulla, -ae (f)}ēbulliet fluvius
rānās: quæ ascendent, et ingredientur domum tuam, et cubiculum lectulī tuī, et super strātum
tuum, et in domōs servōrum tuōrum, et in populum tuum, et in
\mimg{furnus.png}{furnus, -ī (m)\\locus in quō panis coquitur}furnōs tuōs, et in \mpp{reliquiae, -ārum (f)}{pauca illa quae ex aliqua re relicta sunt}reliquiās cibōrum tuōrum:
et ad tē, et ad populum tuum, et ad omnēs servōs tuōs intrābunt
rānæ.''

Dīxitque Dominus ad Moysēn: Dīc ad
Aarōn: ``Extende manum tuam super fluviōs ac super rīvōs et
\mpp{palus, palūdis (f)}{aqua non fluens}palūdēs, et ēdūc rānās super terram Ægyptī.''

Et extendit Aarōn manum super aquās Ægyptī, et ascendērunt
rānæ, operuēruntque terram Ægyptī.  Fēcērunt autem et
maleficī per \mpp{incantātiō, -ōnis (f)}{vis quā maleficus rem mirabilem
facit}incantātiōnēs suās \mpp{similiter:}{$<$
similis}similiter, ēdūxēruntque rānās super terram Ægyptī.
Vocāvit autem Pharaō Moysen et Aarōn, et
dīxit eīs: ``Ōrātē Dominum ut auferat rānās ā mē et ā populō
meō, et dīmittam populum ut \mpp{sacrificare:}{animal interficere vel rem offerre ut placeat deo}sacrificet Dominō.''

Dīxitque Moysēs
ad Pharaōnem: ``Cōnstitue mihi quandō
\mpp{dēprecārī:}{aliquem aut rem aliquam per preces a periculo liberare}dēprecer prō tē, et prō servīs tuīs, et prō populō tuō, ut
\mpp{abigere:}{pellere}abigantur rānæ ā tē, et ā domō tuā, et ā
servīs tuīs, et ā populō tuō: et tantum in flūmine remaneant.''

Quī respondit: ``Crās.''

At ille: ``Iuxtā, inquit, verbum tuum faciam: ut sciās
quoniam nōn est sīcut Dominus Deus noster.  Et recēdent
rānæ ā tē, et ā domō tuā, et ā servīs tuīs, et ā populō tuō:
et tantum in flūmine remanēbunt.''

Ēgressīque sunt
Moysēs et Aarōn ā Pharaōne: et clāmāvit
Moysēs ad Dominum prō \mpp{spōnsiō, -ōnis (f)}{id quod promissum est}spōnsiōne
rānārum quam \mpp{condīcere:}{aliquid inter aliquōs constituere; promittere}condīxerat
Pharaōnī.  Fēcitque Dominus iuxtā verbum Moȳsī: et
mortuæ sunt rānæ dē domibus, et dē vīllīs, et dē agrīs.

\begin{figure}[h]
    \begin{minipage}[h!]{0.5\linewidth}
        \centering
        \includegraphics{agger.png}
        \caption{agger nummōrum}
    \end{minipage}%
    \begin{minipage}[h!]{0.5\linewidth}
        \centering
        \includegraphics{fly}
        \caption{sciniphēs, -um (f pl.)}
    \end{minipage}
\end{figure}

\mpp{congregare:}{in unum locum cogere vel ferre vel ducere}Congregāvēruntque eās in \mpp{immēnsus/a/um:}{tam magnum ut nemo possit pro certo invenire quam magnum sit}immēnsōs
aggerēs, et \mpp{computrescere:}{facere malum odorem (id quod nasus sentit) propter rem mortuam}computruit terra.  Vidēns autem
Pharaō quod data esset \mpp{requies:}{tempus quo homo non laborat}requiēs, \mpp{ingravare:}{facere gravem}ingravāvit cor suum, et nōn
audīvit eōs, sīcut \mpp{praecipere:}{constituere, imperare}præcēperat Dominus. 

Dīxitque Dominus ad Moysēn: ``Loquere ad Aarōn: Extende
virgam tuam, et percute \mpp{pulvis, pulveris:}{terra minima et non umida}pulverem terræ: et sint
sciniphēs in ūniversā terrā Ægyptī.''

Fēcēruntque ita. Et
extendit Aarōn manum, virgam tenēns: percussitque pulverem
terræ, et factī sunt sciniphēs in hominibus, et in
\mpp{iūmentum, -ī (n)}{animal utile ad gerendum vel trahendum: e.g, equī, bovēs, et cetera}iūmentīs: omnis pulvis terræ versus est in
sciniphēs per tōtam terram Ægyptī.  Fēcēruntque
similiter maleficī
incantātiōnibus suīs, ut ēdūcerent
sciniphēs, et nōn potuērunt: erantque
sciniphēs tam in hominibus quam in
iūmentīs.  Et dīxērunt maleficī ad
Pharaōnem: Digitus Deī est hic; \mpp{indurare:}{durum facere}indūrātumque est cor Pharaōnis, et nōn audīvit eōs
sīcut \mpp{praecipere:}{constituere, imperare}præcēperat Dominus.  

Dīxit quoque
Dominus ad Moysēn: \mpp{cōnsurgere:}{surgere}``Cōnsurge
\mpp{dīlūculum, -ī (n)}{prima lux diei}dīlūculō, et stā cōram Pharaōne:
ēgrediētur enim ad aquās: et dīcēs ad eum: Hæc dīcit Dominus: Dīmitte
populum meum ut sacrificet mihi.  Quod sī nōn dīmīserīs eum, ecce ego
\mpp{immitere:}{intus mittere, inducere}immittam in tē, et in servōs tuōs,
et in populum tuum, et in domōs tuās, omne genus \mimg{fly_small.png}{musca, -ae (f)}muscārum:
et implēbuntur domūs Ægyptiōrum muscīs
\mpp{dīversus/a/um:}{in varias partes versi sunt}dīversī generis, et ūniversa terra in quā fuerint. 
Faciamque mīrābilem in diē illa terram Gessen, in quā populus meus est, ut
nōn sint ibi muscæ: et sciās quoniam ego Dominus in mediō
terræ.  Pōnamque \mpp{dīvīsiō, -ōnis (f)}{actus dividendi; id quō res dividitur}dīvīsiōnem inter populum meum et populum
tuum: crās erit signum istud.''

Fēcitque Dominus ita. Et venit
musca gravissima in domōs Pharaōnis et
servōrum eius, et in omnem terram Ægyptī: \mpp{corruptus/a/um:}{id quod in peius mutatur}corruptaque est
terra ab huiuscemodī muscīs.  Vocāvitque
Pharaō Moysen et Aarōn, et ait eīs: ``Īte et
sacrificātē Deō vestrō in terrā hāc.''  

Et ait Moysēs:
``Nōn potest ita fierī: \mpp{abōminātiō, -ōnis (f):}{aliquid pessimum et malum - hīc significat rēs quās Aegyptī putant esse deōs: e.g, bovēs}abōminātiōnēs enim Ægyptiōrum
\mpp{immolāre:}{sacrificium facere}immolābimus Dominō Deō nostrō: quod sī
\mpp{mactāre:}{interficere, occidere}mactāverīmus ea quæ colunt Ægyptiī cōram eīs,
\mimg{rock}{lapis, lapidis (m)}lapidibus nōs \mpp{obruere:}{operīre}obruent.  Viam trium diērum
pergēmus in \mpp{solitudo:}{deserta; locus ubi nemo vel unus
habitat}sōlitūdinem: et sacrificābimus Dominō Deō nostrō, sīcut
\mpp{praecipere:}{imperare}præcēpit nōbīs.''

Dīxitque
Pharaō: ``Ego dīmittam vōs ut 
sacrificētis Dominō Deō vestrō
in dēsertō: \mpp{vērumtamen:}{sed tamen}vērumtamen longius nē abeātis, rogātē prō mē.''

At ait Moysēs: ``Ēgressus ā tē, ōrābō Dominum: et
recēdet musca ā Pharaōne, et ā servīs suīs,
et ā populō eius crās: vērumtamen nōlī ultrā fallere, ut
nōn dīmittās populum 
sacrificāre Dominō.''

Ēgressusque Moysēs ā
Pharaōne, ōrāvit Dominum.  Quī fēcit iuxtā verbum illīus,
et abstulit muscās ā Pharaōne, et ā servīs
suīs, et ā populō eius: nōn superfuit nē ūna quidem.  Et
\mpp{ingravare:}{facere gravem}ingravātum est cor
Pharaōnis, ita ut nec hāc quidem vice dīmitteret populum.
