\section{Quaestiō Augustīnī}

Quaeritur quōmodo populō dīcātur, quod mandāvit Deus ēiectūrum sē eōs dē
Aegyptō in terram Chanaan; Pharaōnī autem dīcātur, quod trium diērum
\mimg{aug.jpg}{Augustinus}
iter exīre vellent in dēsertum immolāre Deō suō ex mandātō eius.

Sed intellegendum est,\marginpar{quamvīs: etsī} quamvīs Deus scīret quid esset factūrus, quoniam
praesciēbat nōn cōnsēnsūrum Pharaōnem ad populum dīmittendum, illud
prīmō dictum esse, quod etiam \marginpar{prīmitus (adv): primum, primo}prīmitus fieret, sī ille dīmitteret. Ut
enim sīc fierent omnia quemadmodum cōnsequēns Scrīptūra testātur,
Pharaōnis \marginpar{contumācia, -ae (f): superbia}contumācia meruit et suōrum. Neque enim mendāciter Deus iubet
quod scit nōn factūrum cui iubētur, ut iūstum iūdicium cōnsequātur. 

\section{Quaestiō Augustīnī Altera}

Verba quae dīcit Moysēs ad Dominum: ``Quārē afflīxistī populum hunc? et
\marginpar{utquid: quamobrem}utquid mē mīsistī? Ex quō enim intrāvī ad Pharaōnem loquī in tuō nōmine,
in hunc populum; et nōn līberāstī populum tuum.'' Nōn contumāciae verba
sunt vel indignātiōnis, sed inquīsītiōnis et ōrātiōnis: quod ex hīs
appāret, quae illī Dominus respondit. Nōn enim arguit īnfidēlitātem
eius, sed quid sit factūrus aperuit.
