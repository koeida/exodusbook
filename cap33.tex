\invisiblechapter{Cap. XXXIII}

\titleimg{ashburn}

\mktitle{Capitulum Trīcēsimum Tertium}
\thispagestyle{empty}

\cstart{L}{ocūtusque} est Dominus ad Moysen, dīcēns: ``Vāde, ascende
dē locō istō tū, et populus tuus quem ēdūxistī dē terrā Ægyptī, in terram
quam \mimg{iuro}{iurāre}iūrāvī Abraham, Isaac et Iācōb, dīcēns:
Sēminī tuō dabō eam: 
\vnum{2}et mittam \mpp{præcursor, -ōris (m)}{qui currit prae aliīs}præcursōrem tuī
\mpp{angelus, -ī (m):}{res viva, sapiens, et potēns ā Deō missa quae agit
quidquid Deus vult}angelum, ut ēiciam Chananæum, et
Amorrhæum, et Hethæum, et Pherezæum, et Hevæum, et Iebusæum, 
\vnum{3}et intrēs in
terram fluentem lacte et melle. Nōn enim ascendam tēcum, quia populus dūræ
\mpp{cervīx, cervīcis (f):}{posterior pars collī; collum}cervīcis es: nē
forte \mpp{disperdere:}{valde perdere}disperdam tē in viā.''

\vnum{4}Audiēnsque
populus sermōnem hunc pessimum, \mpp{lūgeō, lugēre, lūxī, luctum}{}lūxit: et nūllus ex mōre indūtus est \mpp{cultus, -ūs (m)}{ornamenta, vestimenta}cultū
suō. 

\vnum{5}Dīxitque Dominus ad Moysēn: ``Loquere fīliīs Isrāēl:
Populus dūræ cervīcis es: semel ascendam in mediō tuī, et dēlēbō tē. Iam
nunc dēpōne ōrnātum tuum, ut sciam quid faciam tibi.''

\begin{figure}[h!]
    \begin{minipage}[hp]{0.5\linewidth}
        \centering
        \includegraphics{tab2}
        \caption{tabernaculum, -ī (n)}
    \end{minipage}%
    \begin{minipage}[hp]{0.5\linewidth}
        \centering
        \includegraphics{papilio}
        \caption{pāpiliō, -ōnis (m)}
    \end{minipage}
\end{figure}

\vnum{6}Dēposuērunt ergō
fīliī Isrāēl ōrnātum\linebreak suum ā monte Horeb. 
\vnum{7}Moysēs quoque tollēns tabernāculum,
tetendit extrā castra procul, vocāvitque nōmen eius `Tabernāculum fœderis.'
Et omnis populus, quī habēbat aliquam \mpp{quaestiō, -ōnis (f):}{quod
interrogatur, praecipuē cum respondēre difficile est e.g, ``Estne Deus?'',
``Quid est hominibus bonum?'', etc.}quæstiōnem, ēgrediēbātur ad
tabernāculum fœderis, extrā castra. 
\vnum{8}Cumque ēgrederētur Moysēs ad
tabernāculum, surgēbat ūniversa \mpp{plēbs, plēbis (f):}{populī
multitudo}plēbs, et stābat ūnusquisque in ōstiō \mpp{pāpiliō, -ōnis (m):}{tabernaculum quod formam animalis eodem nomine habet}pāpiliōnis
suī, aspiciēbantque tergum Moȳsī, dōnec \mpp{ingredī:}{intus
īre}ingrederētur \mpp{tentōrium, -ī (n):}{tabernaculum}tentōrium. 

\vnum{9}Ingressō autem illō
tabernāculum fœderis, dēscendēbat columna nūbis, et stābat ad ōstium,
loquēbāturque cum Moyse, 
\vnum{10}cernen\-tibus ūniversīs quod
columna nūbis stāret ad ōstium tabernāculī. Stābantque
ipsī, et adōrābant per forēs tabernāculōrum suōrum. 
\vnum{11}Loquēbātur autem Dominus ad Moysēn faciē ad faciem, sīcut solet loquī homō
ad amīcum suum. Cumque ille reverterētur in castra, minister eius Iosve
fīlius Nun, puer, nōn recēdēbat dē
tabernāculō. 

\vnum{12}Dīxit autem Moysēs ad Dominum:
\linebreak ``\mpp{praecipere:}{imperāre}Præcipis ut ēdūcam populum istum: et nōn
\mpp{indicāre:}{mōnstrāre}indicās mihi quem missūrus es mēcum,
\mpp{praesertim:}{praecipue}præsertim cum dīxerīs: Nōvī tē ex nōmine, et
invēnistī grātiam cōram mē. 
\vnum{13}Sī ergō invēnī grātiam in cōnspectū tuō,
ostende mihi faciem tuam, ut sciam tē, et inveniam grātiam ante oculōs tuōs:
\mpp{respicere:}{aspicere iuvandī causā}respice populum tuum gentem hanc.''

\vnum{14}Dīxitque Dominus: ``Faciēs mea \mpp{praecēdere:}{ante īre}præcēdet tē, et
\mpp{requiēs, requiēī (f):}{tempus quō homo non laborat}requiem dabō tibi.''

\vnum{15}Et ait
Moysēs: ``Sī nōn tū ipse præcēdās, nē ēdūcās nōs dē locō istō. 
\vnum{16}In quō
enim scīre poterimus ego et populus tuus invēnisse nōs grātiam in cōnspectū
tuō, nisi ambulāverīs nōbīscum, ut \mpp{glōrificāre:}{gloriosum
facere}glōrificēmur ab omnibus populīs quī habitant super terram?''

\vnum{17}Dīxit
autem Dominus ad Moysēn: ``Et verbum istud, quod locūtus es, faciam:
invēnistī enim grātiam cōram mē, et tēipsum nōvī ex nōmine.''

\vnum{18}Quī ait: ``Ostende mihi glōriam tuam.''

\vnum{19}Respondit: ``Ego ostendam omne bonum tibi, et
vocābō in nōmine Dominī cōram tē: et \mpp{miserērī:}{dolorem sentīre propter miserum alterum}miserēbor cui
voluerō, et clēmēns erō in quem mihi placuerit.''

\vnum{20}\mpp{rūrsum =}{rūrsus}Rūrsumque ait: ``Nōn poteris vidēre faciem meam: nōn enim
vidēbit mē homō et vīvet.'' 

\vnum{21}Et iterum: ``Ecce, inquit, est locus apud mē,
et stābis suprā \mimg{hole}{forāmen, forāminis (n)}\mpp{petra, -ae (f):}{lapis}petram.
\vnum{22}Cumque trānsībit
glōria mea, pōnam tē in forāmine petræ, et
\mpp{prōtegere:}{defendere}prōtegam dexterā meā, dōnec trānseam: 
\vnum{23}tollamque manum
meam, et vidēbis posteriōra mea: faciem autem meam vidēre nōn poteris.''
