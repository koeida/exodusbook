\invisiblechapter{Cap. XIV}

\titleimg{drown}

\mktitle{Capitulum Quārtum Decimum}
\thispagestyle{empty}

\cstart{L}{ocūtus} est autem Dominus ad Moysen, dīcēns: \vnum{2}``Loquēre
fīliīs Isrāēl: Reversī \mpp{castrametari:}{castra
ponere}castramētentur ē regiōne Phihahiroth, quæ est inter Magdalum et mare
contrā Beelsephon: in cōnspectū eius castra pōnētis super mare.
\vnum{3}Dictūrusque est Pharaō super fīliīs Isrāēl:
\mpp{coarctāre:}{in angustum cogere}Coarctātī sunt in terrā;
\mpp{conclūdere:}{claudere}conclūsit eōs
dēsertum. \vnum{4}Et \mpp{indurare:}{durum facere}indūrābō cor eius, ac
persequētur vōs: et \mpp{glōrificāre:}{gloriosum facere}glōrificābor in
Pharaōne, et in omnī exercitū eius; scientque Ægyptiī quia
ego sum Dominus. Fēcēruntque ita.''

\vnum{5}Et nūntiātum est rēgī Ægyptiōrum quod
fūgisset populus: \mpp{immūtātus/a/um:}{non mutatus}immūtātumque est cor\linebreak
Pharaōnis et servōrum eius super populō, et dīxērunt: ``Quid
voluimus facere ut dīmitterēmus Isrāēl, nē servīret nōbīs?''

\vnum{6}Iūnxit ergō
currum, et omnem populum suum \mpp{assūmere:}{sumere ad sē; iungere sibi}assūmpsit sēcum. \vnum{7}Tulitque sexcentōs currūs
ēlēctōs, et quidquid in Ægyptō curruum fuit: et ducēs tōtīus exercitūs.
\vnum{8}Indū\-rāvitque Dominus cor Pharaōnis rēgis Ægyptī, et persecūtus est fīliōs
Isrāēl: at illī ēgressī sunt in manū \mpp{excelsus/a/um:}{altus}excelsā.
\vnum{9}Cumque persequerentur Ægyptiī vestīgia \mpp{praecedere:}{ante
ire}præcēdentium, reperērunt eōs in castrīs super mare: omnis equitātus et
currūs Pharaōnis, et ūniversus exercitus, erant in Phihahiroth contrā
Beelsephon. 

\vnum{10}Cumque appropinquāsset Pharaō, levantēs fīliī Isrāēl oculōs,
vīdērunt Ægyptiōs post sē, et timuērunt valdē: clāmāvēruntque ad Dominum,
\vnum{11}et dīxērunt ad Moysēn: ``Forsitan nōn erant \mpp{sepulchrum, -ī (n):}{locus ubi mortuus positus est}sepulchra in
Ægyptō, ideō tulistī nōs ut morerēmur in \mpp{solitudo, -inis (f):}{deserta; locus ubi
nemo vel unus habitat}sōlitūdine: quid hoc facere voluistī, ut ēdūcerēs
nōs ex Ægyptō? \vnum{12}Nōnne iste est sermō, quem loquēbāmur ad tē in Ægyptō,
dīcentēs: Recēde ā nōbīs, ut serviāmus Ægyptiīs? Multō enim melius erat
servīre eīs, quam morī in sōlitūdine.''

\vnum{13}Et ait Moysēs ad
populum: ``Nōlīte timēre: stāte, et vidēte \mpp{magnālia, -ōrum (n.
pl.):}{res mirandae}magnālia Dominī
quæ factūrus est hodiē: Ægyptiōs enim, quōs nunc vidētis,
\mpp{nequaquam:}{minime, nullo modo}nēquāquam ultrā vidēbitis usque in
\mpp{sempiternus/a/um:}{perpetuus}sempiternum. \vnum{14}Dominus pugnābit prō vōbīs, et
vōs tacēbitis.''

\vnum{15}Dīxitque Dominus ad Moysēn: ``Quid clāmās ad mē? Loquere
fīliīs Isrāēl ut proficīscantur. \vnum{16}Tū autem \mpp{elevare:}{tollere; sursum
levare}ēlevā virgam tuam, et extende manum tuam super mare, et dīvide illud:
ut gradiantur fīliī Isrāēl in mediō marī per siccum. \vnum{17}Ego autem
indūrābō cor Ægyptiōrum ut persequantur vōs: et glōrificābor in Pharaōne,
et in omnī exercitū eius, et in curribus et in equitibus illīus. \vnum{18}Et
scient Ægyptiī quia ego sum Dominus cum glōrificātus fuerō
in Pharaōne, et in curribus atque in equitibus eius.''

\vnum{19}Tollēnsque sē
\mpp{angelus, -ī (m):}{res viva, sapiens, et potēns ā Deō missa quae agit quidquid Deus vult}angelus 
Deī, quī præcēdēbat castra Isrāēl, abiit post eōs:
et cum eō \mpp{pariter:}{aeque}pariter columna nūbis, priōra dīmittēns,
post tergum \vnum{20}stetit, inter castra Ægyptiōrum et castra Isrāēl: et erat
nūbēs \mpp{tenebrōsus/a/um:}{obscurus}tenebrōsa, et \mpp{illūmināre:}{illustrāre}illūmināns noctem, ita
ut ad sē \mpp{invicem:}{Aegyptiī, deinde filiī Isrāel}invicem tōtō noctis tempore accēdere nōn valērent.
\vnum{21}Cumque extendisset Moysēs manum super mare, abstulit illud Dominus
flante ventō \mpp{vehemens:}{ferox, iratus, severus}vehementī et \mpp{ūrere}{id quod ignis agit}ūrente
tōta nocte, et vertit in siccum: dīvīsaque est aqua. \vnum{22}Et
\mpp{ingredi:}{intus ire}ingressī sunt fīliī Isrāēl per medium siccī maris:
erat enim aqua quasi mūrus ā dextrā eōrum et læva. \vnum{23}Persequentēsque
Ægyptiī ingressī sunt post eōs, et omnis equitātus Pharaōnis, currūs eius
et equitēs per medium maris. \vnum{24}Iamque advēnerat vigilia
\mpp{mātūtīnus/a/um:}{quod in mane vel circum mane est}mātūtīna, et ecce \mpp{respicere:}{aspicere iuvandi
causa}respiciēns Dominus super castra Ægyptiōrum per columnam ignis et
nūbis, interfēcit exercitum eōrum, \vnum{25}et \mpp{sub-vertere:}{vertere aliquid ut summa pars ad fundum vertatur et fundus ad summam partem vertatur}subvertit
\mpp{profundus, -ī (m):}{fundus}\mimg{rota}{rota, -ae (f)}rotās curruum, ferēbanturque in profundum.

Dīxērunt ergō Ægyptiī: ``Fugiāmus Israēlem: Dominus enim
pugnat prō eīs contrā nōs.''

\vnum{26}Et ait Dominus ad Moysēn: ``Extende manum tuam
super mare, ut revertantur aquæ ad Ægyptiōs super currūs et equitēs
eōrum.''

\vnum{27}Cumque extendisset Moysēs manum\linebreak contrā mare, reversum est prīmō
\mpp{diluculum, -ī (n):}{prima lux diei}dīlūculō ad priōrem locum: fugientibusque
Ægyptiīs occurrērunt aquæ, et \mpp{involvere:}{facere ut aliquid anguste circumdetur}involvit eōs Dominus in
mediīs flūctibus. \vnum{28}Reversæque sunt aquæ, et operuērunt currūs et equitēs
cūnctī exercitūs Pharaōnis, quī sequentēs ingressī fuerant mare: nec ūnus
quidem superfuit ex eīs. \vnum{29}Fīliī autem Isrāēl perrēxērunt per medium siccī
maris, et aquæ eīs erant quasi prō mūrō ā dextrīs et ā sinistrīs: 
\vnum{30}līberāvitque Dominus in diē illā Isrāēl dē manū Ægyptiōrum. 
\vnum{31}Et vīdērunt
Ægyptiōs mortuōs super \mpp{littus, -ōris (n):}{terra apud mare quae ā fluctibus pulsatur}littus maris, et manum magnam quam
exercuerat Dominus contrā eōs: timuitque populus Dominum, et crēdidērunt
Dominō et Moysī servō eius.
