\chapter{CAPITULUM SECUNDUM}

\marginpar{stirps, stirpis(m/f): stricto sensu est truncus arboris, ergo etiam origo est generis in familiis}Ēgressus est post hæc vir dē domō Levī: et accēpit uxōrem stirpis suæ.
Quæ concēpit, et peperit fīlium : et vidēns eum ēlegantem, abscondit tribus mēnsibus.
\marginpar{scirpeus/a/um: ex scirpīs factus}
\marginpar{linere: rem aliquam liquidam aut mollem alteri superinducere}
\marginpar{bitūmen, -inis(n): limus pinguis et sulphureus ē terrā emergens pluribus locīs}
\marginpar{pix, picis (f): rēs ātra quae fit ex liquōre crassō arborum coctō}
\marginpar{cārex, cāricis (f):  herba acuta et durissima, sparto similis}
\marginpar{cārectum, -ī (n): locus caricibus plenus}

\begin{figure}[hp]
    \begin{minipage}[hbp]{0.5\linewidth}
        \centering
        \includegraphics{fisc}
        \caption{fiscella, -ae(f)}
    \end{minipage}%
    \begin{minipage}[hbp]{0.5\linewidth}
        \centering
        \includegraphics{brush}
        \caption{scirpus, -ī(m)}
    \end{minipage}
\end{figure}

Cumque iam cēlāre nōn posset, sūmpsit fiscellam scirpeam,
et līnīvit eam bitūmine ac pice:
posuitque intus īnfantulum, et exposuit eum in cārectō rīpæ flūminis,
stante procul sorōre eius, et cōnsīderante ēventum reī.
Ecce autem dēscendēbat fīlia Pharaōnis ut lavārētur in flūmine.\marginpar{crepīdo, -inis (f): litus maris vel ripa fluminis, quae saxis, aut lapideo margine plerumque sternebantur}\marginpar{alveus, -ī (m): fossa, per quam fluvius defluit} Et puellæ eius \marginpar{gradī: gradum facere, ambulāre}gradiēbantur per crepīdinem alveī.
\marginpar{papyrio, -ōnis (m): locus papyrīs consitus}
Quæ cum vīdisset fiscellam in papȳriōne,
mīsit ūnam ē famulābus suīs: et allātam aperiēns,
cernēnsque in eā parvulum vāgientem,
miserta eius, ait: ``Dē īnfantibus Hebræōrum est hīc.''

Cui soror puerī: ``Vīs,'' inquit, ``ut vādam, et vōcem tibi mulierem hebræam,
quæ nūtrīre possit īnfantulum?''

Respondit: ``Vāde.''

Perrēxit puella et vocāvit mātrem suam.
Ad quam locūta fīlia Pharaōnis: ``Accipe,'' ait, ``puerum istum, et nūtrī mihi: ego dabō tibi mercēdem tuam.''

Suscēpit mulier, et nūtrīvit puerum: adultumque trādidit fīliæ Pharaōnis. 
Quem illa adoptāvit in locum fīliī,
vocāvitque nōmen eius Moysēs, dīcēns: ``Quia dē aquā tulī eum.''

In diēbus illīs postquam crēverat Moysēs, ēgressus est ad frātrēs suōs:
\marginpar{percutere: pulsare, afficere, et quasi vulnerare dolore}vīditque afflīctiōnem eōrum, et virum ægyptium percutientem quemdam dē Hebræīs frātribus suīs.
Cumque circumspexisset hūc atque illūc,
et nūllum adesse vīdisset,
\marginpar{abscondere: rem aliquam aliquo in loco ponere, ut celetur}
\marginpar{sabulum, -ī: arena}percussum Ægyptium abscondit sabulō.
\marginpar{rixāre: contendere}Et ēgressus diē alterō cōnspexit duōs Hebræōs rixantēs:
dīxitque eī quī faciēbat iniūriam: ``Quārē percutis proximum tuum?''

Quī respondit: ``Quis tē cōnstituit prīncipem et iūdicem super nōs?
num occīdere mē tū vīs, sīcut heri occīdistī Ægyptium?''

Timuit Moysēs, et ait: ``Quōmodo palam factum est verbum istud?''

Audīvitque Pharaō sermōnem hunc, et quærēbat occīdere Moysēn:
quī fugiēns dē cōnspectū eius, morātus est in terrā Madiān,
\marginpar{puteus, -ī (m): aedificium (et foramen) ex quo aqua excipitur}et sēdit iuxtā puteum.
Erant autem sacerdōtī Madian septem fīliæ,
quæ vēnērunt ad hauriendam aquam:
\marginpar{adaquāre: aquam dāre}et implētīs canālibus adaquāre cupiēbant gregēs patris suī.
Supervēnēre pāstōrēs, et ēiēcērunt eās:
surrēxitque Moysēs, et dēfēnsīs puellīs, adaquāvit ovēs eārum. 
Quæ cum revertissent ad Raguel patrem suum, dīxit ad eās:
\marginpar{vēlox, -ōcis: celer}``Cūr vēlōcius vēnistis solitō?''

Respondērunt: ``Vir ægyptius līberāvit nōs dē manū pāstōrum:
\marginpar{īnsuper: praeterea}īnsuper et hausit aquam nōbīscum, pōtumque dedit ovibus.''

At ille: ``Ubi est?'' inquit: ``quārē dīmīsistis hominem? vocātē eum ut comedat pānem.''

Iūrāvit ergō Moysēs quod habitāret cum eō.
Accēpitque Sephoram fīliam eius uxōrem:
quæ peperit eī fīlium, quem vocāvit Gersam, dīcēns:
``Advena fuī in terrā aliēnā.''

Alterum vērō peperit, quem vocāvit Eliezer, dīcēns:
``Deus enim patris meī adiūtor meus ēripuit mē dē manū Pharaōnis.''

Post multum vērō tempore mortuus est rēx Ægyptī:
\marginpar{ingemīscere: dolere; prae animi angustia in sonum prorumpere et queri, suspirare}et ingemīscentēs fīliī Isrāēl
propter opera vōciferātī sunt:
\marginpar{vōciferātus/a/um $<$ vōciferārī: vehementer exclamāre}
ascenditque clāmor eōrum ad Deum ab operibus.
Et audīvit gemitum eōrum,
\marginpar{pangō, pangere, pepigisse, pactum: constituere, definite statuere}ac recordātus est fœderis quod pepigit cum Abraham, Isaac et Iācōb.
Et respexit Dominus fīliōs Isrāēl et cognōvit eōs.
