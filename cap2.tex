\chapter{}

\titleimg{baby2.jpg}

\mktitle{Capitulum Secundum}
\thispagestyle{empty}

Ēgressus est post hæc vir dē domō Levī: et accēpit uxōrem \mpp{stirps, stirpis (m/f):}{truncus arboris; sed hīc significat familiam (fortasse, e.g, filia sororis patris)}stirpis suæ.
Quæ \mpp{concipere:}{femina quae concipit efficit ut infans in sē incipit crescere}concēpit, et peperit fīlium : et vidēns eum \mpp{ēlegāns =}{pulcher}ēlegantem, abscondit tribus mēnsibus.
\mpp{scirpeus/a/um:}{ex scirpīs factus}
\mpp{līnere:}{ponere rem aliquam mollem in alterā premendō ac tergendō}
\mpp{bitūmen, -inis (n):}{māteria mollis atque ātra in terrā inventa quae pōnitur in rēbus nē aqua īnfluere possit}
\mpp{pix, picis (f):}{māteria mollis atque ātra ex parte arboris facta bitūminī similis}
\mpp{cārex, cāricis (f):}{herba acuta et durissima}


\begin{figure}[hp]
    \begin{minipage}[hbp]{0.5\linewidth}
        \centering
        \includegraphics{fisc}
        \caption{fiscella, -ae (f)}
    \end{minipage}%
    \begin{minipage}[hbp]{0.5\linewidth}
        \centering
        \includegraphics{brush}
        \caption{scirpus, -ī (m)}
    \end{minipage}
\end{figure}

Cumque iam cēlāre nōn posset, sūmpsit fiscellam scirpeam,
et līnīvit eam bitūmine ac pice:
posuitque intus īnfantulum, et exposuit eum in \mpp{cārectum, -ī (n):}{locus caricibus plenus}cārectō \mpp{rīpa, -ae (f):}{terra tangēns flumen}rīpæ flūminis,
stante procul sorōre eius, et cōnsīderante \mpp{cōnsīderāns ēventum reī:}{exspectans ut videat quid futurum sit}ēventum reī.
Ecce autem dēscendēbat fīlia Pharaōnis ut lavārētur in flūmine.\mpp{crepīdo, -inis (f):}{ripa fluminis lapidibus strata}\mpp{alveus, -ī (m):}{fossa, per quam fluvius fluit} Et puellæ eius \mpp{gradī:}{gradum facere, ambulāre}gradiēbantur per crepīdinem alveī.
\mpp{papyrio, -ōnis (m):}{locus multīs cum papyrīs}
Quæ cum vīdisset fiscellam in papȳriōne,
mīsit ūnam ē \mpp{famula, -ae = }{ancilla}famulābus suīs: et allātam aperiēns,
cernēnsque in eā parvulum vāgientem,
\mpp{miserta (+ gen.):}{tristis propter aliquem miserum}miserta eius, ait: ``Dē īnfantibus Hebræōrum est hīc.''

Cui soror puerī: ``Vīs,'' inquit, ``ut vādam, et vōcem tibi mulierem hebræam,
quæ \mpp{nūtrīre:}{dāre lac}nūtrīre possit \mpp{īnfantulus/a/um:}{parvus īnfans}īnfantulum?''

Respondit: ``Vāde.''

Perrēxit puella et vocāvit mātrem suam.
Ad quam locūta fīlia Pharaōnis: ``Accipe,'' ait, ``puerum istum, et nūtrī mihi: ego dabō tibi mercēdem tuam.''

\mpp{suscipere:}{sumere, capere; subtus accipere aut prehendere}Suscēpit mulier, et nūtrīvit puerum: \mpp{adultus/a/um:}{iam vir/femina, non puer/puella}adultumque trādidit fīliæ Pharaōnis. 
\mpp{adoptāre:}{qui aliquem adoptat efficit ut filia vel filius suus fiat}Quem illa adoptāvit in locum fīliī,
vocāvitque nōmen eius Moysēs, dīcēns: ``Quia dē aquā tulī eum.''

In diēbus illīs postquam crēverat Moysēs, ēgressus est ad frātrēs suōs:
\mpp{percutere:}{pulsare, afficere, et quasi vulnerare dolore}vīditque afflīctiōnem eōrum, et virum ægyptium percutientem quemdam dē Hebræīs frātribus suīs.
Cumque circumspexisset hūc atque illūc,
et nūllum adesse vīdisset,
\mimg{sabulum.png}{sabulum, -ī (n)}\mpp{abscondere:}{rem aliquam aliquo in loco ponere, ut celetur}
percussum Ægyptium abscondit sabulō.
\mpp{rixāre:}{certāre}Et ēgressus diē alterō cōnspexit duōs Hebræōs rixantēs:
dīxitque eī quī faciēbat iniūriam: ``Quārē percutis proximum tuum?''

\mpp{prīnceps, prīncipis (m):}{rex; dux}Quī respondit: ``Quis tē cōnstituit prīncipem et \mimg{judge_small}{\hspace*{0.75cm}iūdex, iūdicis (m/f)}iūdicem super nōs?
num occīdere mē tū vīs, sīcut heri occīdistī Ægyptium?''

\mpp{palam}{$\leftrightarrow$ celatum}Timuit Moysēs, et ait: ``Quōmodo palam factum est verbum istud?''

Audīvitque Pharaō sermōnem hunc, et quærēbat occīdere Moysēn:
quī fugiēns dē cōnspectū eius, morātus est in terrā Madiān,
\mimg{puteus.png}{puteus, -ī (m)}et sēdit iuxtā puteum.
Erant autem sacerdōtī Madian septem fīliæ,
quæ vēnērunt ad \mpp{haurīre:}{sumere ex loco ubi aqua est}hauriendam aquam:
\mpp{adaquāre:}{aquam dāre}et implētīs canālibus adaquāre cupiēbant gregēs patris suī.
\mpp{supervēnīre:}{subitō adesse nocendī causā}\mpp{supervēnēre =}{supervēnērunt}Supervēnēre pāstōrēs, et ēiēcērunt eās:
surrēxitque Moysēs, et dēfēnsīs puellīs, adaquāvit ovēs eārum. 
Quæ cum revertissent ad Raguel patrem suum, dīxit ad eās:
\mpp{vēlox, -ōcis:}{celer}``Cūr vēlōcius vēnistis solitō?''

Respondērunt: ``Vir ægyptius līberāvit nōs dē manū pāstōrum:
\mpp{īnsuper:}{praeterea}īnsuper et hausit aquam nōbīscum, \mpp{pōtus, -ūs (m) =}{pōtiō}pōtumque dedit ovibus.''

At ille: ``Ubi est?'' inquit: ``quārē dīmīsistis hominem? vocātē eum ut \mpp{comedere:}{totum ēsse; ēsse}comedat pānem.''

Iūrāvit ergō Moysēs quod habitāret cum eō.
Accēpitque Sephoram fīliam eius uxōrem:
quæ peperit eī fīlium, quem vocāvit Gersam, dīcēns:
\mpp{advena, -ae (m)}{quī nōn est cīvis, qui ab aliō locō venit}``Advena fuī in \mpp{terra aliēna:}{terra ubi is nōn natus est}terrā aliēnā.''

Alterum vērō peperit, quem vocāvit Eliezer, dīcēns:
\mpp{adiūtor, -ōris (m):}{qui adiuvat}``Deus enim patris meī adiūtor meus ēripuit mē dē manū Pharaōnis.''

Post multum vērō tempore mortuus est rēx Ægyptī:
\mpp{ingemīscere:}{dolēre}et ingemīscentēs fīliī Isrāēl
propter opera vōciferātī sunt:
\mpp{vōciferātus/a/um $<$ vōciferārī:}{valde exclamāre}
ascenditque clāmor eōrum ad Deum ab operibus.
\mpp{gemitus, -ūs (m):}{sonus atque actus dolendī}Et audīvit gemitum eōrum,
\mpp{pangō, pangere, pepigisse, pactum:}{constituere, statuere}ac \mpp{recordor, -ārī, -ātus sum =}{meminisse}recordātus est fœderis quod pepigit cum Abraham, Isaac et Iācōb.
\mpp{respicere:}{aspicere iuvandī causā}Et respexit Dominus fīliōs Isrāēl et cognōvit eōs.
