\invisiblechapter{Cap. XXI-XXIV}

\titleimg{sinai}

\mktitle{Capitula ā Vīcēsimō Prīmō}
\vspace*{-1.4cm}
\mktitle{ad Vīcēsimum Quartum}
\thispagestyle{empty}

[Permultae Deī lēgēs in capitulīs ā vīcēsimō prīmō ad vīcēsimum tertium
continentur. Multae sunt lēgēs dē iniūriīs: e.g, ``Quī percusserit patrem
suum aut mātrem, morte moriātur.''

Multae sunt lēgēs dē agrīs et animālium
gregibus: e.g, ``Sī occurrerīs bovī inimīcī tuī, aut asinō errantī, redūc
ad eum.''

Multae sunt lēgēs dē Deō colendō: e.g, \mpp{prīmitiae, -ārum (f):}{primae rēs quae ex agrīs metuntur}``Prīmitiās frūgum terræ
tuæ \mpp{dēferre:}{ferre aliquid aliquō}dēferēs in domum Dominī Deī tuī.''

Hīc incipit
capitulum vīcēsimum quārtum:]

\mimg{root}{rādīx, rādīcis (f)}\vnum{1}[Deus] Moȳsī quoque dīxit: ``Ascende ad Dominum
tū, et Aarōn, Nadab et Abiu, et septuāgintā senēs ex
Isrāēl, et adōrābitis procul.
\vnum{2}Solusque
Moysēs ascendet ad Dominum, et illī nōn appropinquābunt:
nec populus ascendet cum eō.''

\vnum{3}Venit ergō Moysēs et nārrāvit
\mpp{plēbs, plēbis}{populī multitudo}plēbī omnia verba Dominī, atque
\mimg{judge_small}{iudex, iudicis (m/f)}\mpp{iūdicium, -ī:}{quod iudex tradit}iūdicia:
responditque omnis populus ūnā vōce: ``Omnia verba Dominī, quæ locūtus est,
faciēmus.''

\vnum{4}Scrīpsit autem Moysēs ūniversōs sermōnēs Dominī: et manē
\mpp{cōnsurgere:}{surgere}cōnsurgēns, ædificāvit altāre ad
\mpp{rādīx montis $\leftrightarrow$}{altissima pars montis}rādīcēs montis,
et duodecim \mpp{titulus, -ī (m):}{columna}titulōs per duodecim tribus
Isrāēl. 

\vnum{5}Mīsitque iuvenēs dē fīliīs Isrāēl, et obtulērunt
\mpp{holocaustum, -ī (n):}{sacrificium igne factum}holocausta,
\mpp{immolāre:}{sacrificium facere}immolāvēruntque \mpp{victima, -ae
(f):}{animal sacrificiō statutum}victimās \mpp{pācificus/a/um:}{pacem
faciens}pācificās Dominō, \mpp{vitulus, -ī (m):}{bōs parvus}vitulōs.  
\vnum{6}Tulit itaque Moysēs
dīmidiam partem sanguinis, et mīsit in \mimg{crater}{crātēr, crātēris (m)}crātērās: partem
autem \mpp{residuus/a/um:}{quod reliquum est}residuam fūdit super altāre. 

\vnum{7}Assūmēnsque \mpp{volūmen, volūminis (n):}{liber}volūmen fœderis, legit audiente populō: quī
dīxērunt: ``Omnia quæ locūtus est Dominus, faciēmus, et erimus
\mpp{obēdiēns, -entis (adj):}{parēns}obēdientēs.''

\vnum{8}Ille vērō sūmptum sanguinem
\mpp{respergere:}{aspergere}respersit in populum, et ait: ``Hic est sanguis fœderis quod
\mpp{pangō, pangere, pepigī, pactum:}{constituere}pepigit Dominus vōbīscum super cūnctīs sermōnibus hīs.''

\vnum{9}Ascendēruntque Moysēs et Aarōn, Nadab et Abiu, et septuāgintā dē seniōribus
Isrāēl: 
\vnum{10}et vīdērunt Deum Isrāēl: et sub pedibus eius quasi opus
\mimg{rock}{lapis, lapidis (m)}lapidis \mimg{bluegem}{sapphīrus, -ī (m)}\mpp{sapphīrinus/a/um <}{sapphīrus}sapphīrinī, et quasi cælum, cum serēnum est. 
\vnum{11}Nec
super eōs quī procul recesserant dē fīliīs Isrāēl, mīsit manum suam,
vīdēruntque Deum, et \mpp{comedere:}{totum ēsse; ēsse}comēdērunt, ac
bibērunt. 

\vnum{12}Dīxit autem Dominus ad Moysēn: ``Ascende ad mē
in montem, et estō ibi: dabōque tibi tabulās \mpp{lapideus/a/um:}{ex
lapidibus factum}lapideās, et lēgem, ac \mpp{mandātum, -ī (n):}{id quod
imperatum vel traditum est}mandāta quæ scrīpsī ut doceās eōs.''

\vnum{13}Surrēxērunt Moysēs et Iōsve minister eius: ascendēnsque Moysēs in montem
Deī, 
\vnum{14}seniōribus ait: ``\mpp{expectāre =}{exspectāre}Expectātē hīc dōnec revertāmur ad
vōs. Habētis Aarōn et Hur vōbīscum: sī quid nātum fuerit
\mpp{quaestiō, -ōnis (f):}{quod interrogatur, praecipuē cum respondēre difficile est: e.g, ``Estne Deus?'', ``Quid est hominibus bonum?'', etc.}quæstiōnis, referētis ad eōs.''

\vnum{15}Cumque ascendisset Moysēs,
operuit nūbēs montem, 
\vnum{16}et habitāvit glōria Dominī super Sinaī, tegēns
illum nūbe sex diēbus: septimō autem diē vocāvit eum dē mediō
\mpp{cālīgō, cālīginis (f):}{aer per quem difficile est vidēre}cālīginis. 
\vnum{17}Erat autem speciēs glōriæ Dominī quasi ignis
\mpp{ārdēre:}{habēre ignem in sē}ārdēns super \mpp{vertex, verticis
(m):}{altissima pars}verticem montis in cōnspectū fīliōrum Isrāēl. 
\vnum{18}\mpp{ingredī:}{intus īre}Ingressusque Moysēs medium \mpp{nebula, -ae (f):}{nūbēs prope terram}nebulæ,
ascendit in montem: et fuit ibi quadrāgintā diēbus, et quadrāgintā
noctibus.
