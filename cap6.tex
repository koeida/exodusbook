\chapter{}
\mktitle{Capitulum Sextum}

Dīxitque Dominus ad Moysēn: ``Nunc vidēbis quæ factūrus
sim Pharaōnī: per manum enim fortem dīmittet
\marginpar{rōbustus/a/um: durus, rigidus, valens}eōs, et in manū rōbustā ēiiciet illōs
dē terrā suā.''

Locūtusque est Dominus ad Moysēn dīcēns: ``Ego Dominus
quī appāruī Abraham, Isaac et Iācōb
in Deō omnipotente: et nōmen meum Adonaī
\marginpar{pangō, pangere, pepigisse, pactum: constituere, definite statuere}\marginpar{foedus, foederis (n): res inter duōbus hominēs vel civitatēs constituta} nōn indicāvī eīs. Pepigīque fœdus cum eīs, ut
darem eīs terram Chanaan, terram peregrīnātiōnis
eōrum, in quā fuērunt advenæ. 
Ego audīvī gemitum fīliōrum Isrāēl, quō
Ægyptiī oppressērunt eōs: et recordātus
sum pactī meī. Ideō dīc fīliīs Isrāēl: Ego Dominus
\marginpar{ergastulum, -ī: carcer rusticus}quī ēdūcam vōs dē ergastulō Ægyptiōrum, et
ēruam dē servitūte, ac redimam in brāchiō
excelsō et jūdiciīs magnīs. Et assūmam vōs mihi
in populum, et erō vester Deus: et sciētis quod ego
sum Dominus Deus vester quī ēdūxerim vōs dē
ergastulō Ægyptiōrum, et indūxerim in terram, super quam
levāvī manum meam ut darem eam Abraham, Isaac
et Iācōb: dabōque illam vōbīs possidendam. 
Ego Dominus.''

Nārrāvit ergō Moysēs omnia fīliīs Isrāēl: quī nōn
\marginpar{acquiescere: assentiri, fidem habere}acquiēvērunt eī
propter angustiam spīritūs, et opus dūrissimum. Locūtusque est
Dominus ad Moysen, dīcēns: ``Ingredere, et loquere ad Pharaōnem rēgem
Ægyptī, ut dīmittat fīliōs Isrāēl dē terrā suā.''

Respondit Moysēs cōram Dominō: ``Ecce fīliī Isrāēl nōn audiunt mē: et
quōmodo audiet
\marginpar{``incircumcīsus labiīs'' fortasse sibi vult labia (id est,
facultas loquendi) non iam deo serviendo apta esse, sicut vir
incircumcisus non iam deo serviendo aptus est} Pharaō, præsertim cum incircumcīsus sim labiīs?'' 

Locūtusque est
Dominus ad Moysēn et Aarōn, et dēdit mandātum ad fīliōs Isrāēl, et ad
Pharaōnem rēgem Ægyptī ut ēdūcerent fīliōs Isrāēl dē terrā Ægyptī.
