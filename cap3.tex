\chapter{}

\titleimg{burning_bush.jpg}

{\begin{center}\large\bf\underline{Capitulum Tertium}\end{center}%}
\vspace*{-0.75cm}
Moysēs autem pāscēbat ovēs Iethrō socerī suī sacerdōtis Madian:
\marginpar{mināre: cōgere animālia}cumque mināsset gregem ad interiōra dēsertī,
venit ad montem Deī Horeb.
\marginpar{rubus, -ī (m): quaedam planta spinosa}
Appāruitque eī Dominus in flammā ignis dē mediō rubī:
\marginpar{ārdēre: igne flagrāre}\marginpar{combūrere: igne cōnsūmere}et vidēbat quod rubus ārdēret, et nōn combūrerētur.
Dīxit ergō Moysēs : ``Vādam, et vidēbō vīsiōnem hanc magnam, quārē nōn combūrātur rubus.''

Cernēns autem Dominus quod pergeret ad videndum,
vocāvit eum dē mediō rubī, et ait: ``Moysēs, Moysēs.''

Quī respondit: ``Adsum.''

\marginpar{calceāmentum, -ī: quod pedī induitur}At ille: ``Nē appropiēs,'' inquit, ``hūc: solve calceāmentum dē pedibus tuīs: locus enim,
in quō stās, terra sāncta est.'' 

Et ait: ``Ego sum Deus patris tuī, Deus Abraham, Deus Isaac et Deus Iācōb.''

Abscondit Moysēs faciem suam: nōn\linebreak enim audēbat aspicere contrā Deum.

\marginpar{afflīctiō, -ōnis (f): rēs quae valde displicet, vexat, nocet}Cui ait Dominus: ``Vīdī afflīctiōnem populī meī in Ægyptō,
\marginpar{{\bf duritia, -ae (f)} $<$ durus/a/um}et clāmōrem eius audīvī propter dūritiam eōrum quī præsunt operibus:
et sciēns dolōrem eius, dēscendī ut libērem eum dē manibus Ægyptiōrum,
\marginpar{spatiōsus/a/um: multum spatiī habēns}et ēdūcam dē terrā illā in terram bonam, et spatiōsam,
\marginpar{lac, lactis (n): quod infantēs bibunt}in terram quæ fluit lacte et melle,
ad loca Chananæī et Hethæī, et Amorrhæī, et Pherezæī, et Hevæī, et Iebusæī.
Clāmor ergō fīliōrum Isrāēl venit ad mē: vīdīque afflīctiōnem eōrum,
qua ab Ægyptiīs opprimuntur.
Sed vēnī, et mittam tē ad Pharaōnem,
ut ēducās populum meum, fīliōs Isrāēl, dē Ægyptō.''

Dīxitque Moysēs ad Deum: ``Quis sum ego ut vādam ad Pharaōnem,
et ēdūcam fīliōs Isrāēl dē Ægyptō?''

Quī dīxit eī: ``Ego erō tēcum: et hoc habēbis signum,
quod mīserim tē: cum ēdūxerīs populum meum dē Ægyptō,
immolābis Deō super montem istum.''

Ait Moysēs ad Deum: ``Ecce ego vādam ad fīliōs Isrāēl,
et dīcam eīs: Deus patrum vestrōrum mīsit mē ad vōs.
Sī dīxerint mihi: `Quod est nōmen eius?' quid dīcam eīs?''

Dīxit Deus ad Moysēn: ``EGO SUM QUĪ SUM.'' Ait: ``Sīc dīcēs fīliīs Isrāēl: `QUĪ EST mīsit mē ad vōs.' ''

Dīxitque iterum Deus ad Moysēn: ``Hæc dīcēs fīliīs Isrāēl:
Dominus Deus patrum vestrōrum, Deus Abraham, Deus Isaac et Deus Iācōb,
\marginpar{in æternum: semper} mīsit mē ad vōs: hoc nōmen mihi est in æternum,
\marginpar{memoriāle, -is (n): id quod memorandum est}et hoc memoriāle meum in generātiōnem et generātiōnem.''

``Vāde, et congregā seniōrēs Isrāēl,
et dīcēs ad eōs: Dominus Deus patrum vestrōrum appāruit mihi,
Deus Abraham, Deus Isaac et Deus Iācōb,
\marginpar{vīsitāns vīsītāvī: vīsitāvī (rhētoricē paene idem iterum dicit)}dīcēns: Vīsitāns vīsitāvī vōs: et vīdī omnia quæ accidērunt vōbīs in Ægyptō.
Et dīxī ut ēdūcam vōs dē afflīctiōne Ægyptī in terram Chananæī,
et Hethæī, et Amorrhæī, et Pherezæī, et Hevæī, et Iebusæī,
ad terram fluentem lacte et melle.''

``Et audient vōcem tuam: ingrediērisque tū,
et seniōrēs Isrāēl, ad rēgem Ægyptī, et dīcēs ad eum:
Dominus Deus Hebræōrum vocāvit nōs:
\marginpar{sōlitūdo, -inis $<$ solus/a/um: regio deserta; locus ubi nemo vel unus habitat}ībimus viam trium diērum in sōlitūdinem,
\marginpar{immolāre: sacrificium facere}ut immolēmus Dominō Deō nostrō.''

``Sed ego sciō quod nōn dīmittet vōs rēx Ægyptī
ut eātis nisi per manum validam.
Extendam enim manum meam, et percutiam Ægyptum
in cūnctīs mīrābilibus meīs, 
quæ factūrus sum in mediō eōrum:
post hæc dīmittet vōs.
Dabōque grātiam populō huic cōram Ægyptiīs:
\marginpar{vacuī: id est, sine rēbus}et cum ēgrediēminī, nōn exībitis vacuī:
\marginpar{vīcīnus/a: qui prope habitat}sed postulābit mulier ā vīcīnā suā
et ab hospitā suā, vāsa argentea et aurea, ac vestēs:
\marginpar{spoliāre: capere rēs aliōrum hominum}pōnētisque eās super fīliōs et fīliās vestrās, et spoliābitis Ægyptum.''
