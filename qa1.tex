\section{Quaestiō Augustīnī}

{\it Dē obstetrīcum mendāciō, quō fefellērunt Pharaōnem, nē
occīderent masculōs Isrā-ēlītās quandō nāscēbantur, dīcentēs
nōn ita pārēre mulierēs Hebraēās sīcut pariēbant Aegyptiae:}

\mimg{aug.jpg}{Augustinus}Quaerī solet utrum tālia mendācia approbāta sint
auctōritāte dīvīnā, quandōquidem scrīptum est Deum
bene fēcisse obstetrīcibus: sed utrum prō misericordiā
ignōscēbat mendāciō; an et ipsum mendācium dignum
praemiō iūdicābat, incertum est.

\marginpar{vīvificāre: vivum facere, vitam dare}Aliud enim faciēbant obstetrīcēs vīvificandō
īnfantēs parvulōs, aliud Pharaōnī mentiendō: nam
in illīs vīvificandīs opus misericordiae fuit; mendāciō
vērō illō prō sē ūtēbantur, nē nocēret illīs
Pharaō, quod potuit nōn ad laudem, sed ad veniam pertinēre.

\marginpar{id est, ut ita dīcam, haec fābula nōn dat licentiam mentiendī}Neque hinc auctōritātem ad mentiendum esse prōpositam
\marginpar{eōrum: sanctōrum}mihi vidētur eīs dē quibus dictum est: ``Et nōn est
inventum in ōre eōrum mendācium.''

Quōrumdam enim vīta longē
īnferior ā professiōne sānctōrum, sī habeat ista
mendāciōrum peccāta, prōvectū ipsō et indole feruntur,
\marginpar{{\bf nōrunt} = nōvērunt}praesertim sī beneficia dīvīna nōndum nōrunt exspectāre
coelestia, sed circā terrēna occupantur.

\marginpar{conversātiō, -ōnis (f): modus vīvendī}Quī autem ita vīvunt, ut eōrum conversātiō,
sīcut dīcit Apostolus,
in coelīs sit, nōn eōs exīstimō linguae suae modum,
\marginpar{non exīstimō eōs debēre formāre modum linguae suae exemplō illō obstetrīcum}
quantum ad vēritātem prōmendam attinet falsitātemque vītandam,
exemplō illō obstetrīcum dēbēre fōrmāre.

\marginpar{disserere: sermōnem īnstituere dē rē aliquā, dīcere, disputāre, tractāre}Sed dīligentius dē hāc quaestiōne disserendum est,
propter alia exempla quae in Scrīptūrīs reperiuntur.
