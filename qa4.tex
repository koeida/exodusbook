\section{Quaestiō Augustīnī}

Quemadmodum possit intellegī īrāscēns\linebreak
Deus, quia nōn sīcut homō per irratiōnābilem perturbātiōnem, per omnia
tenendum est, ubi tāle aliquid Scrīptūra dīcit, nē dē hōc eadem saepe
\mimg{aug.jpg}{Augustinus}
dīcenda sint. Sed meritō quaeritur cūr hīc īrātus dē frātre Moyse
dīxerit, quod ipse illī loquerētur ad populum: vidētur enim tamquam
\marginpar{diffīdere: nōn fīdere}diffīdentī nōn dedisse plēnissimam facultātem, quam datūrus erat; et per 
duōs agī voluisse, quod et per ūnum posset, sī crēdidisset.

Vērumtamen
eadem verba omnia dīligentius cōnsīderāta, nōn significant īrātum
\marginpar{vindicta, -ae (f): ultio vel animadversio pro delicto;
vindicatio}Dominum prō vindictā dedisse Aarōn. Sīc enim dīcit: ``Nōnne ecce Aarōn
frāter tuus Lēvītēs? sciō quia loquēns loquētur ipse.''

Quibus verbīs
ostenditur Deus increpāsse potius eum, quī timēret īre quod ipse esset
minus idōneus, cum habēret frātrem per quem posset ad populum loquī quod
\marginpar{gracilis, -is (adj): tenuis, exilis}vellet; quoniam erat ipse gracilis vōcis, et linguae tardiōris: quamquam
dē Deō tōtum spērāre dēbēret. Deinde eadem ipsa quae paulō ante
prōmīserat, et posteāquam īrātus est, dīcit. Dīxerat enim: ``Aperiam os
tuum, et īnstruam tē''; nunc autem dīcit: ``Aperiam os tuum et os eius, et
īnstruam vōs quae faciātis''; sed quoniam addidit: ``Et loquētur ipse tibi
ad populum,'' vidētur ōris apertiō praestita, propter quod dīcit Moysēs
linguae sē esse tardiōris. 

Dē vōcis autem gracilitāte nihil eī praestāre
\marginpar{adiūtōrium, -ī (n): auxilium, quodqumque adiuvat}Dominus voluit, sed propter hoc adiūtōrium frātris adiūnxit, quī posset
eā utī vōce, quae pōpulō docendō sufficeret. Quod ergō ait: ``Et dabis
verba mea in os eius,'' ostendit quod ea loquenda esset datūrus: nam sī
\marginpar{tantummodo: solum, solummodo}tantummodo audienda, sīcut populō, in aurēs dīceret. Deinde quod paulō
post ait: ``Et loquētur ipse tibi ad populum, et ipse erit tuum os, et hic
subaudītur, ad populum.'' Et cum dīcit: ``Tibi loquētur ad populum''; satis
indicat in Moysēn prīncipātum, in Aarōn ministerium. Deinde quod ait: ``Tū
\marginpar{fortassis = fortasse}autem illī eris quae ad Deum,'' magnum
hīc fortassis perscrūtandum 
\marginpar{sacrāmentum, -ī: mysterium, signum sensibile rei sacrae
latentis}est sacrāmentum, cuius figūram gerat, velutī medius Moysēs inter Deum et
Aarōn, et medius Aarōn inter Moysēn et populum. 

\section{Quaestiō Augustīnī Altera}

\marginpar{obviāre: obviam īre}In eō quod scrīptum est: Et factum est, in viā ad refectiōnem obviāvit
\marginpar{calculus: parvus lapis}eī angelus, et quaerēbat eum occīdere: et assūmptō Sepphora calculō,
\marginpar{praepūtium, -ī (n): illa pars puerī quae circumcīditur}circumcīdit praepūtium fīliī suī; et prōcidit ad pedēs eius, et dīxit:
``Stetit sanguis circumcīsiōnis īnfantis meī. Et recessit ab eō;'' propter
quod dīxit: ``Dēsiit sanguis circumcīsiōnis;'' 

Prīmum quaeritur, quem
volēbat angelus occīdere, utrum Moysen, quia dictum est, ``occurrit eī
angelus, et quaerēbat eum occīdere.'' Nam cui putābitur occurrisse, nisi
\marginpar{comitātus, -ūs (m): comitum multitudo}illī quī ūniversō suōrum comitātuī praefuit, et ā quō caeterī
dūcēbantur? An puerum quaerēbat occīdere, cui māter circumcīdendō
\marginpar{subvenīre: opem ferre, auxiliari, succurrere}subvenit; ut ob hoc intellegātur occīdere voluisse īnfantem, quia nōn
\marginpar{sancīre: aliquid ritu sacro, seu religioso decernere,
constituere}erat circumcīsus, atque ita sancīre praeceptum circumcīsiōnis,
sevēritāte vindictae?

Quod sī ita est, incertum est prius dē quō
\marginpar{id est, ut ita dicam, quis est ``eum''? Quis est ``eī''?}dīxerit, ``quaerēbat eum occīdere;'' quia ignōrātur quem, nisi ex
cōnsequentibus reperiātur: mīrā sānē locūtiōne et inūsitātā, ut prius
dīceret, ``occurrit eī,'' et quaerēbat eum occīdere, dē quō nihil anteā
dīxerat.

Sed tālis est in Psalmō: ``Fundāmenta eius in montibus sānctīs;
dīligit Dominus portās Sīōn.'' Inde enim Psalmus incipit, nec aliquid
dē illō vel dē illā dīxerat, cuius fundāmenta intellegī voluit, dīcēns:
Fundāmenta eius in montibus sānctīs. Sed quia sequitur, ``dīligit Dominus
portās Sīōn,'' ergō fundāmenta vel Dominī vel Sīōn, et ad faciliōrem
sēnsum magis Sīōn, ut fundāmenta cīvitātis accipiantur. Sed quia in hōc
prōnōmine, quod est, ``eius,'' genus ambiguum est (omnis enim generis est
hoc prōnōmen, id est et masculīnī, et fēminīnī, et neutrī), in graecō
autem in fēminīnō genere ``autes'' dīcātur, masculīnō et neutrō
``autou'', et habet cōdex
graecus ``autou'', cōgit intellegere nōn fundāmenta Sīōn, sed fundāmenta Dominī,
id est, quae cōnstituit Dominus, dē quō dictum est: ``Aedificāns Ierusālem
Dominus.'' Nec Sīōn tamen, nec Dominum anteā nōmināverat, cum dīceret:
``Fundāmenta eius in montibus sānctīs.'' 

Sīc et hīc nōndum nōminātō īnfante
dictum est, ``occurrit eī, et quaerēbat eum occīdere;'' ut dē quō dīxerit,
in cōnsequentibus agnōscāmus.  Quamquam etsī dē Moyse accipere quisquam
\marginpar{magnopere: valde, vehementer}voluerit, nōn est magnopere resistendum.

Illud potius quod sequitur, sī
fierī potest, intellegātur, quid sibi velit ideō recessisse angelum ab
interfectiōne cuiuslibet eōrum, quia dīxit mulier: ``Stetit sanguis
circumcīsiōnis īnfantis.'' Nōn enim ait: Recessit ab eō, propter quod
circumcīdit īnfantem: sed quia stetit sanguis circumcīsiōnis; nōn quia
cucurrit, sed quia stetit: magnō, nisi fallor, sacrāmentō. 
