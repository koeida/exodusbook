\section{Quaestiō Augustīnī}

\mimg{aug.jpg}{Augustinus}{\it Dē factō Moȳsī, cum occīdit Aegyptium ad dēfendendōs
frātrēs suōs: utrum indolēs in eō laudābilis fuerit,
\marginpar{admittere: permittere}\marginpar{ūber, ūberis (n): fertilitas, fecunditas}
\marginpar{ferācitas, -ātis (f): fertilitas, fecunditas}quā hoc peccātum admīserit, sīcut solet ūber terrae,
\marginpar{id est, vidimus terram herbārum inūtilium plēnam et dicimus, ``Ecce! Tam fertilis terra!'' -- res mala (herbae inūtiles) est signum reī bonae (fertilitatis)}etiam ante ūtilia sēmina, quādam herbārum quamvīs
inūtilium ferācitāte laudārī;
an omnīnō ipsum factum iustificandum sit.}

\marginpar{lēgitimus/a/um: qui secundum leges est}Quod ideō nōn vidētur, quia nūllam adhūc lēgitimam
\marginpar{dīvīnitus (adv): ā Deō}potestātem gerēbat, nec acceptam dīvīnitus, nec
hūmānā societāte ōrdinātam.

Tamen, sīcut Stephanus
dīcit in Āctibus Apostolōrum, putābat intellegere
frātrēs suōs, quod per eum Deus daret illīs
salūtem: ut per hoc testimōnium videātur
Moysēs iam dīvīnitus admonitus (quod
Scrīptūra eō locō tacet) hoc audēre potuisse.
