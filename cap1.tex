\chapter{}

%\begin{figure}[p]
%    \hspace*{-2.5cm}
%    \makebox[\linewidth]{
%        \setlength{\fboxsep}{0pt}
%        \fbox{\includegraphics[width=1.3\linewidth]{midwives}}
%    }
%    \caption{\hspace*{-4.5cm}Obstetricēs et Rēx}
%\end{figure}

\titleimg{mw2.jpg}

\mktitle{Capitulum Primum}

    [In Aegyptō] fīliī Isrāēl crēvērunt, et \mpp{quasi germinantēs:}{non vero germinantēs (quia hominēs non germinant) sed vidērī aliquō modō germināre}quasi\mpp{germināre:}{emittere germina (ea quae ex plantīs veniunt: semen, fructus, cēt.)} germinantēs \mpp{multiplicāre:}{multum facere; augēre}multiplicātī sunt:
ac rōborātī \mpp{rōborāre:}{virēs dāre}nimis, implēvērunt terram.

Surrēxit intereā rēx novus super Ægyptum, quī ignōrābat Ioseph.
Et ait ad populum suum: ``Ecce, populus fīliōrum Isrāēl multus, et fortior nōbīs est.
Venīte, \mpp{sapienter (adv) <}{sapiēns}\mpp{opprimere:}{efficere ut aliquis surgere nōn potest}\mpp{ingruere:}{cum vi accedere}sapienter opprimāmus eum, nē forte multiplicētur: et sī ingruerit contrā nōs bellum, addātur inimīcīs nostrīs, expugnātisque nōbīs ēgrediātur dē terrā.''

\begin{figure}[h]
    \begin{minipage}[h]{0.5\linewidth}
        \centering
        \includegraphics{tab}
        \caption{tabernaculum, -ī (n)}
    \end{minipage}%
    \begin{minipage}[h]{0.5\linewidth}
        \vspace*{-0.3cm}
        \centering
        \includegraphics{later}
        \vspace*{-0.425cm}
        \caption{later, lateris (m)}
    \end{minipage}
\end{figure}

Præposuit \mpp{praeposuit eīs magistrōs operum:}{dedit magistrīs potestatem super opera fīliōrum Isrāēl}itaque eīs magistrōs operum, ut afflīgerent eōs \mpp{onus, oneris (n):}{res quam difficile est portare vel agere. e.g: saccus lapidibus plenus, labor difficilis} oneribus: 
ædificaveruntquē urbēs tabernāculōrum Pharaōnī, Phithom et Ramessēs.
Quantōque opprimēbant eōs, tantō magis multiplicābantur, et crēscēbant: 
ōderantque fīliōs Isrāēl Ægyptiī, et afflīgēbant \mpp{illūdere:}{deridēre}illūdentēs eīs,
atque ad \mpp{amāritūdo, -inis (f)}{\\< amārus/a/um $\leftrightarrow$ dulcis, -e}amāritūdinem \mpp{per-dūcere:}{trahere}perdūcēbant vītam eōrum operibus dūrīs \mpp{lutum, -ī (n):}{terra humida}lutī et lateris, omnīque \mpp{famulātus, -ūs (m):}{servitus; conditiō servōrum}famulātū, quō in terræ operibus premēbantur. 
Dīxit autem rēx Ægyptī \mpp{obstetrix, -īcis (f):}{femina quae iuvat feminam gravidam parere} obstetrīcibus Hebræōrum, quārum ūna vocābātur Sephora, altera Phua, 
\mpp{praecipere:}{imperāre} præcipiēns eīs: ``Quandō obstetrīcābitis Hebræās, et \mpp{partus, -ūs (m):}{actus pariendi} partus tempus advēnerit: sī masculus fuerit, interficite eum: sī fēmina, \mpp{reservāre:}{servāre, conservāre}reservātē.''

Timuērunt autem obstetrīcēs Deum, \mpp{praeceptum, -ī:}{quod praecipitur}\mpp{nōn fēcērunt iuxtā praeceptum regis:}{non parent praeceptō rēgis}et nōn fēcērunt iuxtā præceptum rēgis Ægyptī, sed \mpp{cōnservāre:}{servāre}cōnservābant \mpp{mas, maris (m):}{homo (vel animal) masculinus}marēs. 
Quibus ad sē accersītis, rēx ait: ``Quidnam est hoc quod facere voluistis, ut puerōs servārētis?''

Quæ respondērunt: ``Nōn sunt Hebreæ sīcut Ægyptiæ mulierēs: ipsæ enim obstetrīcandī habent \mpp{scientia, -ae (f):}{id quod scītur}scientiam, et priusquam veniāmus ad eās, pariunt.''

Bene ergō fēcit Deus obstetrīcibus: et crēvit populus, \mpp{cōnfortāre:}{fortem facere, consolārī}cōnfortātusque est nimis.
Et quia timuērunt obstetrīcēs Deum, ædificāvit eīs domōs.
Præcēpit ergō Pharaō omnī populō suō, dīcēns: ``Quidquid masculīnī sexūs nātum fuerit, in flūmen prōiicite: quidquid fēminīnī, reservātē.''
