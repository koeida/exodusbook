\vnum{1}Vidēns autem populus quod moram faceret dēscendendī dē monte
Moysēs, \mpp{congregāre:}{in unum locum cōgere vel ferre
vel ducere}congregātus adversus Aarōn, dīxit: ``Surge, fac nōbīs deōs, quī
nōs \mpp{praecēdere:}{ante īre}præcēdant: Moysī enim huic
virō, quī nōs ēdūxit dē terrā Ægyptī, ignōrāmus quid acciderit.''

\vnum{2}Dīxitque
ad eōs Aarōn: ``Tollite \mpp{inaurēs, inaurium (f. pl.):}{ornamenta in auribus posita}inaurēs aureās dē uxōrum,
fīliōrumque et fīliārum vestrārum auribus, et afferte ad mē.''

\vnum{3}Fēcitque
populus quæ iusserat, \mpp{dēferre:}{ferre aliquid aliquō}dēferēns inaurēs
ad Aarōn. 
\vnum{4}Quās cum ille accēpisset, \mpp{formāre:}{formam alicui reī dāre, facere}fōrmāvit opere \mpp{fūsōrius/a/um:}{quod funditur}fūsōriō, et fēcit ex
eīs \mpp{vitulus, -ī (m):}{bōs parvus}vitulum \mpp{cōnflātilis, -e (adj):}{quod flandō factum est}cōnflātilem. Dīxēruntque: ``Hī
sunt dīī tuī Isrāēl, quī tē ēdūxērunt dē terrā Ægyptī.''

\vnum{5}Quod cum vīdisset Aarōn, ædificāvit altāre cōram eō, et
\mpp{præcō, præcōnis (m):}{quī magnā voce aliquid nuntiat}præcōnis vōce clāmāvit dīcēns: Crās \mpp{sōlemnitās, -ātis
(f):}{diēs quī certīs temporibus quotannis fit quō homines deum
serviunt}sōlemnitās Dominī est. 
\vnum{6}Surgen\-tēsque māne, obtulērunt
\mpp{holocaustum, -ī (n):}{sacrificium igne factum}holocausta, et \mpp{hostia, -ae (f):}{animal quod in sacrificiīs interficitur}hostiās
\mpp{pācificus/a/um:}{pacem faciens}pācificās, et sēdit populus
\mpp{mandūcāre:}{cibum dentibus iterum iterumque premere}mandūcāre, et bibere, et surrēxērunt lūdere. 

\vnum{7}Locūtus est
autem Dominus ad Moysen, dīcēns: ``Vāde, dēscende:
\mpp{peccātum, -ī:}{res contra legem acta}peccāvit populus tuus, quem
ēdūxistī dē terrā Ægyptī. 
\vnum{8}Recessērunt cito dē viā, quam ostendistī eīs:
fēcēruntque sibi vitulum cōnflātilem, et adōrāvērunt, atque
\mpp{immolāre:}{sacrificium facere}immolantēs eī hostiās, dīxērunt: Istī
sunt dīī tuī Isrāēl, quī tē ēdūxērunt dē terrā Ægyptī.'' 

\vnum{9}\mpp{rūrsum =}{rūrsus}Rūrsumque ait Dominus ad Moysēn:\linebreak ``Cernō quod populus iste
dūræ \mpp{cervīx, cervīcis (f):}{posterior pars collī; collum}cervīcis sit: 
\vnum{10}dīmitte mē, ut īrāscātur \mpp{furōr,
-ōris (m):}{qui habet `furōrem' in sē valde perturbatur}furor meus contrā
eōs, et dēleam eōs, faciamque tē in gentem magnam.'' 

\vnum{11}Moysēs autem ōrābat Dominum Deum suum, dīcēns: ``Cūr, Domine, īrāscitur furor tuus contrā
populum tuum, quem ēdūxistī dē terrā Ægyptī, in \mpp{fortitudō, -inis (f) <}{fortis}fortitūdine magnā, et in manū
\mpp{rōbustus/a/um:}{durus, valens}rōbustā? 
\vnum{12}Nē, \mpp{quæsō:}{homo dicit `quæsō' cum velit verba sua minus duriora viderī}quæsō, dīcant Ægyptiī: 
\mpp{callidus/a/um:}{doctus, praecipue rebus agendis, non rebus ex librīs}Callidē ēdūxit eōs,
ut interficeret in montibus, et dēlēret ē terrā: quiescat īra tua, et estō
\mpp{plācābilis, -e (adj):}{qui facile tranquillus fiērī potest}plācābilis
super \mpp{nēquitia, -ae (f):}{e.g, homo saepe mentitur: hoc est nēquitia eius. Homo saepe iratus est: hoc est nēquitia eius}nēquitiā populī tuī. 
\vnum{13}\mpp{recordor, -ārī, -ātus sum:}{meminisse}Recordāre Abraham, Isaac, et
Isrāēl servōrum tuōrum, quibus \mimg{iuro}{iurāre}iūrāstī per tēmetipsum,
dīcēns: \mpp{multiplicāre:}{multum facere; augēre}Multiplicābō sēmen
vestrum sīcut stēllās cælī; et ūniversam terram hanc, dē quā locūtus sum,
dabō sēminī vestrō, et possidēbitis eam semper.''

\vnum{14}Plācātusque est Dominus
nē faceret mālum quod locūtus fuerat adversus populum suum. 
\vnum{15}Et reversus
est Moysēs dē monte, portāns duās tabulās \mpp{testimōnium, -ī (n):}{id
quod ā teste dīcitur}testimōniī in manū suā, scrīptās ex utrāque parte, 
\vnum{16}et factās opere Deī: \mpp{scrīptūra, -ae (f):}{verba scrīpta}scrīptūra quoque Deī erat \mimg{pump}{rēs \emph{sculpta}}sculpta in
tabulīs. 
\vnum{17}Audiēns autem Iōsve tumultum populī \mpp{vociferārī:}{valde
exclamare}vōciferantis, dīxit ad Moysēn: ``Ululātus pugnæ audītur in
castrīs.'' 

\vnum{18}Quī respondit: ``Nōn est clāmor \mpp{adhortārī:}{hortārī}adhortantium ad
pugnam, neque \mpp{vōciferātiō, -ōnis (f):}{clamor}vōciferātiō \mpp{compellere:}{vī facere ut aliquis aliquid agat}compellentium ad
fugam: sed vōcem cantantium ego audiō.''

\vnum{19}Cumque appropinquāsset ad
castra, vīdit vitulum, et \mpp{chorus, -ī (m):}{actus corporis movendi
propter hominem cantantem vel similem}chorōs: īrātusque valdē, prōiēcit dē
manū tabulās, et \mpp{cōnfringere:}{ūnā frangere}cōnfrēgit eās ad
rādīcem montis: 
\vnum{20}\mpp{arripere:}{vī
prehendere}arripiēnsque vitulum quem fēcerant, \mpp{comburō, comburere, combussī, combussum:}{accendere}combussit,
et \mpp{conterō, conterere, contrīvī, contrītum:}{perdere}contrīvit usque
ad \mpp{pulvis, pulveris (m):}{terra minima et non umida}pulverem, quem sparsit
\mimg{root}{rādīx, rādīcis (f)}in aquam, et dedit ex eō pōtum fīliīs Isrāēl. 

\vnum{21}Dīxitque ad Aarōn: ``Quid
tibi fēcit hic populus, ut \mpp{inducere:}{ducere in aliquem locum}indūcerēs super eum peccātum maximum?''

\vnum{22}Cui
ille respondit: ``Nē \mpp{indignārī:}{aliquid graviter animō ferre; ferre non posse}indignētur dominus meus: tū enim nōstī
populum istum, quod prōnus sit ad mālum. 
\vnum{23}Dīxērunt mihi: Fac nōbīs
deōs, quī nōs præcēdant: huic enim Moȳsī, quī nōs ēdūxit dē terrā Ægyptī,
nescīmus quid acciderit. 
\vnum{24}Quibus ego dīxī: Quis vestrum habet aurum?
Tulērunt, et dedērunt mihi: et prōiēcī illud in ignem, ēgressusque est hic
\mpp{vitulus, -ī (m):}{bōs parvus}vitulus.'' 

\vnum{25}Vidēns ergō Moysēs populum
quod esset nūdātus (spoliāverat enim eum Aarōn\linebreak propter
\mpp{ignōminia, -ae (f):}{id quod minime decet}\mpp{ignōminia sordis:}{id est, vitulus a populō factus}ignōminiam sordis, et inter hostēs nūdum cōnstituerat), 
\vnum{26}et stāns in portā castrōrum, ait: ``Sī quis est Dominī, iungātur mihi.''

Congregātīque sunt ad eum omnēs fīliī Levī 
\vnum{27}quibus ait: ``Hæc dīcit
Dominus Deus Isrāēl: Pōnat vir gladium super \mimg{femur}{femur, femoris/feminis (n)}femur suum. Īte, et reditē
dē portā usque ad portam per medium castrōrum, et occīdat ūnusquisque
frātrem, et amīcum, et proximum suum.''

\vnum{28}Fēcēruntque fīliī Levī iuxtā
sermōnem Moȳsī, cecidēruntque in diē illā quasi vīgintī tria mīllia
hominum. 

\vnum{29}Et ait Moysēs: ``\mpp{cōnsecrāre:}{sacrum facere}Cōnsecrāstis
manūs vestrās hodiē Dominō, \mpp{ūnusquisque:}{singulī hominēs}ūnusquisque in fīliō, et in frātre suō, ut
dētur vōbīs \mpp{benedictiō, -ōnis (f) <}{benedicere: sanctum facere}benedictiō.''

\vnum{30}Factō autem alterō diē, locūtus
est Moysēs ad populum: ``\mpp{peccāre:}{contra legem agere}Peccāstis \mpp{peccatum, -i (n):}{res contra legem acta}peccātum maximum: ascendam ad Dominum,
sī quōmodo \mpp{queō, quīre, quīvī, quitum:}{posse}quīverō eum \mpp{dēprecārī:}{aliquem aut aliquid per preces
a periculo liberare}dēprecārī prō scelere vestrō.''

\vnum{31}Reversusque ad
Dominum, ait: ``\mpp{obsecrāre:}{orāre, precārī}Obsec\-rō, peccāvit populus
iste peccātum maximum, fēcēruntque sibi deōs aureōs: aut dīmitte eīs hanc
\mpp{noxa, -ae (f):}{quod nocet}noxam, 
\vnum{32}aut sī nōn facis, dēlē mē dē librō tuō quem
scrīpsistī.'' 

\vnum{33}Cui respondit Dominus: ``Quī peccāverit mihi, dēlēbō eum dē
librō meō: 
\vnum{34}tū autem vāde, et dūc populum istum quō locūtus sum tibi:
\mpp{angelus, -ī (m):}{res viva, sapiens, et potēns ā Deō missa quae agit
quidquid Deus vult}angelus meus præcēdet tē. Ego autem in diē
\mpp{ulciscī:}{e.g, Agricola pastorem necat, deinde amicus pastoris agricolam necat, id est, amicus pastoris `ulciscitur' agricolam.}\mpp{ultiō, -ōnis (f) <}{ulciscī}ultiōnis \mpp{vīsitāre:}{vīsum īre}vīsitābō et hoc peccātum
eōrum.''

\vnum{35}Percussit ergō Dominus populum prō \mpp{reātus, -ūs (m):}{peccatum propter quod homo puniendus est}reātū vitulī,
quem fēcerat Aarōn. 
