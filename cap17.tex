\chapter{}

\titleimg{arms.jpg}

\mktitle{Capitulum Septimum Decimum}
\thispagestyle{empty}


\cstart{I}{gitur} profecta omnis multitūdō fīliōrum Isrāēl dē
dēsertō Sīn per \mpp{mānsiō, -ōnis (f):}{locus ubi iter facientes noctu manent}mānsiōnēs suās, iuxtā sermōnem Dominī,
\mpp{castramētārī:}{castra ponere}castramētātī sunt in Raphidim, ubi nōn
erat aqua ad bibendum populō. 

\vnum{2}Quī \mpp{iurgārī:}{querī}iūrgātus contrā
Moysēn, ait: ``Dā nōbīs aquam, ut bibāmus.''

Quibus respondit Moysēs: ``Quid iūrgāminī contrā mē? Cūr
\mpp{tentāre:}{conārī aliquid ut videatur quid fiat}tentātis Dominum?''

\vnum{3}Sitīvit ergō ibi populus præ aquæ \mpp{pēnūria, ae (f):}{id quod paene minus sit quam necesse est}pēnūriā, et
\mpp{murmurāre:}{loqui de re quae non placet}murmurāvit contrā Moysen,
dīcēns: Cūr fēcistī nōs exīre dē Ægyptō, ut occīderēs nōs, et līberōs
nostrōs, ac \mpp{iūmentum, -ī (n):}{animal utile ad gerendum vel trahendum e.g, equī, bovēs, et cetera}iūmenta sitī? 

\vnum{4}Clāmāvit autem Moysēs ad
Dominum, dīcēns: ``Quid faciam populō huic? Adhūc \mpp{paululum =}{paulum}paululum,
et \mimg{rock}{lapis, lapidis (m)}\mpp{lapidāre:}{lapidēs iacere}lapidābit mē.''

\vnum{5}Et ait Dominus ad Moysēn:
\mpp{antecēdere:}{ante īre}``Antecēde populum, et sūme tēcum dē seniōribus Isrāēl: et
virgam quā percussistī fluvium, tolle in manū tuā, et vāde. 
\vnum{6}Ēn ego stābō ibi cōram tē, suprā \mpp{petra, -ae (f):}{lapis}petram
Horeb: percutiēsque petram, et exībit ex eā aquā, ut bibat
populus.''

Fēcit Moysēs ita cōram seniōribus Isrāēl 
\vnum{7}et vocāvit nōmen locī
illīus, \mpp{tentātiō, -ōnis (f) <}{tentāre}Tentātiō, propter \mpp{iūrgium, -ī (n) <}{iūrgārī}iūrgium fīliōrum
Isrāēl, et quia tentāvērunt Dominum, dīcentēs: ``Estne Dominus in nōbīs, an
nōn?''

\vnum{8}Venit autem Amalec, et pugnābat contrā Isrāēl in Raphidim.
\vnum{9}Dīxitque Moysēs ad Iōsue: ``Ēlige virōs: et ēgressus, pugnā contrā Amalec:
crās ego stābō in \mpp{vertex, verticis (m):}{altissima pars}vertice collis, habēns virgam Deī in manū
meā.''

\vnum{10}Fēcit Iōsue ut locūtus erat Moysēs, et pugnāvit contrā Amalec:
Moysēs autem et Aarōn et Hur ascendērunt super verticem
collis. 
\vnum{11}Cumque levāret Moysēs manūs, vincēbat Isrāēl: sīn autem
paululum remīsisset, \mpp{superāre:}{vincere}superābat Amalec. 
\vnum{12}Manūs autem Moȳsī
erant gravēs: sūmentēs igitur \mimg{rock}{lapis, lapidis (m)}lapidem, posuērunt
\mpp{subter $\leftrightarrow$}{super}subter eum, in quō sēdit: Aarōn autem et Hur
\mpp{sustentāre:}{sustinēre}sustentābant manūs eius ex utrāque parte. Et factum est ut
manūs illīus nōn \mpp{lassāre:}{fatigāre}lassārentur usque ad occāsum sōlis. 
\vnum{13}Fugāvitque Iōsue Amalec, et populum eius in ōre gladiī. 

\vnum{14}Dīxit autem
Dominus ad Moysēn: ``Scrībe hoc ob \mpp{monimentum, -ī (n):}{quod nos
monet}monimentum in librō, et trāde auribus Iōsue: dēlēbō enim memoriam
Amalec sub cælō.'' 

\vnum{15}Ædificavitque Moysēs altāre: et vocāvit nōmen eius
`Dominus \mpp{exaltātiō, -ōnis (f) < }{exaltāre (in altum levāre)}exaltātiō mea' dīcēns: 
\vnum{16}``Quia manus soliī Dominī,
et bellum Dominī erit contrā Amalec, ā \mpp{generātio, -ōnis (f):}{e.g una
generatio parentibus constat, līberīs eōrum altera generatio constat,
etc...}generātiōne in generātiōnem.''
