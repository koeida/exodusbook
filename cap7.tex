\chapter{}

\titleimg{blood.jpg}

{\begin{center}\large\bf\underline{Capitulum Septimum}\end{center}%}

Dīxitque Dominus ad Moysēn: ``Ecce cōnstituī tē Deum Pharaōnis: et Aarōn
frāter tuus erit prophēta tuus. Tū loqueris eī omnia quæ mandō tibi:
et ille loquētur ad Pharaōnem, ut dīmittat fīliōs Isrāēl dē terrā suā. 
Sed ego indūrābō cor eius, et multiplicābō signa et ostenta mea in terrā
Ægyptī, et nōn audiet vōs: immittamque manum meam super Ægyptum, et
ēdūcam exercitum et populum meum fīliōs Isrāēl dē terrā Ægyptī per
iūdicia maxima. Et scient Ægyptiī quia ego sum Dominus quī extenderim
manum meam super Ægyptum, et ēdūxerim fīliōs Isrāēl dē mediō eōrum.''

Fēcit itaque Moysēs et Aarōn sīcut præcēperat Dominus: ita ēgerunt. 
Erat autem Moysēs octōgintā annōrum, et Aarōn octōgintā trium, quandō
locūtī sunt ad Pharaōnem.

Dīxitque Dominus ad Moysēn et Aarōn: ``Cum dīxerit vōbīs Pharaō,
Ostendite signa: dīcēs ad Aarōn: Tolle virgam tuam, et prōiice eam
\marginpar{coluber, -brī (m): serpens}cōram Pharaōne, ac vertētur in
colubrum.''

Ingressī itaque Moysēs et
Aarōn ad Pharaōnem, fēcērunt sīcut præcēperat Dominus: tulitque Aarōn
virgam cōram Pharaōne et servīs eius, quæ versa est in colubrum.
\marginpar{maleficus, -ī (m): magus, incantator}Vocāvit autem Pharaō sapientēs et maleficōs: et fēcērunt etiam ipsī per
incantātiōnēs ægyptiacās et arcāna quædam similiter. Prōiēcēruntque
\marginpar{dracō, -ōnis (m): serpens}singulī virgās suās, quæ versæ sunt in dracōnēs: sed dēvorāvit virga
Aarōn virgās eōrum. Indūrātumque est cor Pharaōnis, et nōn audīvit
eōs, sīcut præcēperat Dominus.

\marginpar{ingravāre: facere gravem}Dīxit autem Dominus ad Moysēn: ``Ingravātum est cor Pharaōnis: nōn vult
dīmittere populum. Vade ad eum māne, ecce ēgrediētur ad aquās: et
\marginpar{occursus, -ūs (m): obviam itio, actus occurendī}stābis in occursum eius super rīpam flūminis: et virgam quæ conversa
est in dracōnem, tollēs in manū tuā.  Dīcēsque ad eum: Dominus Deus
Hebræōrum mīsit mē ad tē, dīcēns: Dīmitte populum meum ut sacrificet
mihi in dēsertō: et usque ad præsēns audīre nōluistī. Hæc igitur
dīcit Dominus: In hōc sciēs quod sim Dominus: ecce percutiam virgā,
quæ in manū meā est, aquam flūminis, et vertētur in sanguinem. Piscēs
\marginpar{computrēscere: facere odorem malum propter res mortuas}quoque, quī sunt in fluviō, morientur, et computrēscent aquæ, et
afflīgentur Ægyptiī bibentēs aquam flūminis.''

Dīxit quoque Dominus ad
Moysēn: ``Dīc ad Aarōn: Tolle virgam tuam, et extende manum tuam super
\marginpar{palus, palūdis (f): locus plenum aquae quae non fluit; aqua
stagnans}aquās Ægyptī, et super fluviōs eōrum, et rīvōs ac palūdēs, et omnēs
lacūs aquārum, ut vertantur in sanguinem: et sit cruor in omnī terrā
Ægyptī, tam in ligneīs vāsīs quam in saxeīs.''

Fēcēruntque Moysēs et
Aarōn sīcut præcēperat Dominus: et ēlevāns virgam percussit aquam
flūminis cōram Pharaōne et servīs eius: quæ versa est in sanguinem. 
Et piscēs, quī erant in flūmine, mortuī sunt: computruitque fluvius, et
nōn poterant Ægyptiī bibere aquam flūminis, et fuit sanguis in tōtā
terrā Ægyptī. 

Fēcēruntque similiter maleficī Ægyptiōrum
incantātiōnibus suīs: et indūrātum est cor Pharaōnis, nec audīvit eōs,
\mimg{fodit}{homō fodit}
sīcut præcēperat Dominus. Āvertitque sē, et ingressus est domum suam,
\marginpar{hāc vice: hāc occasione; nunc temporis}nec apposuit cor etiam hāc vice. Fōdērunt autem omnēs Ægyptiī per
circuitum flūminis aquam ut biberent: nōn enim poterant bibere dē aquā
flūminis. Implētīque sunt septem diēs, postquam percussit Dominus
fluvium.

