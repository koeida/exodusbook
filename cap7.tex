\chapter{}

\titleimg{blood.jpg}

\mktitle{Capitulum Septimum}
\thispagestyle{empty}

Dīxitque Dominus ad Moysēn: ``Ecce cōnstituī tē Deum
\mpp{Pharaō, -ōnis (m):}{rēx Aegyptōrum}Pharaōnis: et Aarōn frāter tuus erit
\mpp{prophēta, -ae (m):}{homo sanctus qui bene deum intellegit et de deo
loquitur}prophēta tuus. Tū loqueris eī omnia quæ
\mpp{mandāre:}{tradere}mandō tibi: et ille loquētur ad Pharaōnem,
ut dīmittat fīliōs Isrāēl dē terrā suā. Sed ego
\mpp{indūrāre:}{durum facere}indūrābō cor eius, et
\mpp{multiplicare:}{multum facere; augere}multiplicābō signa et ostenta mea
in terrā Ægyptī, et nōn audiet vōs: \mpp{immitere:}{intus mittere,
inducere}immittamque manum meam super Ægyptum, et ēdūcam exercitum et
populum meum fīliōs Isrāēl dē terrā Ægyptī per
\mimg{judge}{iudex, iudicis (m/f)}\mpp{iūdicium, -ī (n):}{quod iudex tradit}iūdicia maxima. Et scient Ægyptiī quia ego sum Dominus
quī extenderim manum meam super Ægyptum, et ēdūxerim fīliōs
Isrāēl dē mediō eōrum.''

Fēcit itaque
Moysēs et Aarōn sīcut \mpp{praecipere:}{imperāre}præcēperat
Dominus: ita ēgerunt. Erat autem Moysēs octōgintā
annōrum, et Aarōn octōgintā trium, quandō locūtī sunt ad
Pharaōnem.

Dīxitque Dominus ad Moysēn et
Aarōn: ``Cum dīxerit vōbīs Pharaō, ostendite signa: dīcēs
ad Aarōn: Tolle virgam tuam, et prōiice eam cōram
Pharaōne, ac vertētur in \mimg{snake}{coluber, -brī (m)}colubrum.''

Ingressī \mpp{ingredī:}{intus īre}itaque Moysēs et Aarōn ad Pharaōnem, fēcērunt sīcut
\mpp{praecipere:}{imperāre}præcēperat Dominus: tulitque Aarōn virgam cōram
Pharaōne et servīs eius, quæ versa est in
colubrum. Vocāvit autem Pharaō sapientēs
et \mpp{maleficus, -ī (m):}{homo quī rēs mirabilēs facere potest}maleficōs et fēcērunt etiam ipsī per
\mpp{incantātiō, -ōnis (f)}{vis quā maleficus rem mirabilem
facit}incantātiōnēs Aegyptiacās et
\mpp{arcānum, -ī (n):}{rēs occulta}arcāna quaedam \mpp{similiter (adv)}{$<$ similis}similiter. Prōiēcēruntque
singulī virgās suās, quae versae sunt in \mpp{dracō, -ōnis (m):}{coluber}dracōnēs: sed
dēvorāvit virga Aarōn virgās eōrum.  \mpp{indūrāre:}{durum facere}Indūrātumque est cor Pharaōnis, et nōn audīvit eōs,
sīcut \mpp{praecipere:}{imperāre}præcēperat Dominus.

Dīxit autem Dominus
ad Moysēn: \mpp{ingravāre:}{facere gravem}``Ingravātum est cor
Pharaōnis: nōn vult dīmittere populum. Vade ad eum
māne, ecce ēgrediētur ad aquās: et \mpp{stābis in occursum eius:}{stābis in loco
ubi occurrēs eī}stābis in occursum eius super rīpam
flūminis: et virgam quæ conversa est in dracōnem, tollēs
in manū tuā. Dīcēsque ad eum: Dominus Deus Hebræōrum
mīsit mē ad tē, dīcēns: Dīmitte populum meum ut sacrificet
mihi in dēsertō: et usque ad præsēns audīre nōluistī. Hæc igitur dīcit
Dominus: In hōc sciēs quod sim Dominus: ecce percutiam virgā, quæ in manū
meā est, aquam flūminis, et vertētur in sanguinem.  Piscēs quoque, quī
sunt in fluviō, morientur, et \mpp{computrēscere:}{facere malum odorem (id quod
nasus sentit) propter rem mortuam}computrēscent aquæ, et
\mpp{afflīgere:}{perdere}afflīgentur Ægyptiī bibentēs aquam flūminis.''

Dīxit
quoque Dominus ad Moysēn: ``Dīc ad Aarōn: Tolle virgam
tuam, et extende manum tuam super aquās Ægyptī, et super fluviōs eōrum, et
rīvōs ac \mpp{palūs, palūdis (f):}{aqua nōn fluēns}palūdēs, et omnēs lacūs aquārum, ut vertantur in
sanguinem: et sit cruor in omnī terrā Ægyptī, tam in ligneīs vāsīs quam in
\mpp{saxum, -ī (n):}{lapis}saxeīs.''

Fēcēruntque Moysēs et Aarōn
sīcut \mpp{praecipere:}{imperāre}præcēperat Dominus: et
\mpp{ēlevāre:}{tollere; sursum levāre}ēlevāns virgam percussit aquam flūminis cōram
Pharaōne et servīs eius: quæ versa est in sanguinem.  Et
piscēs, quī erant in flūmine, mortuī sunt: computruitque
fluvius, et nōn poterant Ægyptiī bibere aquam flūminis, et fuit sanguis in
tōtā terrā Ægyptī.  Fēcēruntque similiter maleficī
Ægyptiōrum incantātiōnibus suīs: et \mpp{indurare:}{durum
facere}indūrātum est cor Pharaōnis, nec audīvit eōs, sīcut
\mpp{praecipere:}{constituere; imperāre}præcēperat Dominus.  Āvertitque sē, et
ingressus est domum suam, nec apposuit cor etiam hāc vice.
 Fōdērunt autem omnēs Ægyptiī per \mpp{circuitus, -ūs (m):}{locus circum
 alterum locum}circuitum flūminis
aquam ut biberent: nōn enim poterant bibere dē aquā flūminis. 
Implētīque sunt septem diēs, postquam percussit Dominus fluvium.
