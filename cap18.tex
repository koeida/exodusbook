\chapter{}

\titleimg{arms.jpg}

\mktitle{Capitulum Duodevicesimum}
\thispagestyle{empty}

\cstart{C}{umque} audīsset Iethrō, \mpp{sacerdōs, -ōtis (m):}{quī res sacrās agit vel qui ducit aliōs in rēbus sacrīs agendīs}sacerdōs Mad\-ian, \mpp{cognātus/a/um (adj):}{qui pars eiusdem familiae est -- Iethrō est pater uxoris Moysī}cognātus Moȳsī,
omnia quæ fēcerat Deus Moȳsī, et Israēlī populō suō, et
quod ēdūxisset Dominus Isrāēl dē Ægyptō, 
\vnum{2}tulit Sephoram
uxōrem Moȳsī quam remīserat, 
\vnum{3}et duōs fīliōs eius: quōrum ūnus vocābātur
Gersam, dīcente patre: \mpp{advena, -ae (m/f):}{quī non civis est, quī in terrā aliā natus est}``Advena fuī in terrā aliēnā;'' 
\vnum{4}alter vērō Eliezer:
``Deus enim,'' ait, ``patris meī \mpp{adiūtor, -ōris (m):}{qui adiuvat}adiūtor
meus, et ēruit mē dē gladiō Pharaōnis.'' 

\vnum{5}Venit ergō Iethrō
cognātus Moȳsī, et fīliī eius, et uxor eius ad Moysēn in
dēsertum, ubi erat \mpp{castramētārī:}{castra ponere}castramētātus iuxtā
montem Deī. 
\vnum{6}Et \mpp{mandātum, -ī (n):}{id quod imperatum vel traditum
est}mandāvit Moysī, dīcēns: ``Ego Iethrō cognātus tuus veniō
ad tē, et uxor tua, et duo fīliī cum eā.'' 

\vnum{7}Quī ēgressus in \mpp{occursus, -ūs (m) <}{occurere}occursum
cognātī suī, adōrāvit, et ōsculātus est eum:
salūtāvēruntque sē \mpp{mūtuō salūtāvērunt:}{Hic illum salūtāvit et ille hunc salūtāvit}mūtuō verbīs \mpp{pācificus/a/um:}{pacem faciens}pācificīs. Cumque intrāsset
\mimg{tab2}{tabernaculum, -ī (n)}tabernāculum, 
\vnum{8}nārrāvit Moysēs
cognātō suō cūncta quæ fēcerat Dominus
Pharaōnī et Ægyptiīs propter Isrāēl: ūniversumque labōrem,
quī accidisset eīs in itinere, et quod līberāverat eōs Dominus.
\vnum{9}Lætatusquē est Iethrō super omnibus bonīs, quæ fēcerat Dominus Israēlī, eō
quod ēruisset eum dē manū Ægyptiōrum. 

\vnum{10}Et ait: ``\mpp{benedīcere:}{sanctum facere}Benedictus Dominus, quī līberāvit vōs dē manū Ægyptiōrum, et dē manū
Pharaōnis; quī ēruit populum suum dē manū Ægyptī. 
\vnum{11}Nunc cognōvī, quia
magnus Dominus super omnēs deōs: eō quod superbē ēgerint contrā illōs.''

\vnum{12}Obtulit ergō Iethrō cognātus Moȳsī \mpp{holocaustum, -ī (n):}{sacrificium igne
factum}holocausta et \mpp{hostia, -ae (f):}{animal quod in sacrificiīs interficitur}hostiās Deō: vēnēruntque Aarōn et omnēs seniōrēs
Isrāēl, ut \mpp{comedere:}{totum ēsse; ēsse}comederent pānem cum eō cōram
Deō. 
\vnum{13}Alterā autem diē sēdit Moysēs ut \mimg{judge_small}{iudex, iudicis (m/f)}\mpp{iudicāre:}{quod iudex
agit}iūdicāret populum, quī \mpp{assistere:}{stāre ad aliquid vel prope aliquid}assistēbat Moȳsī ā māne usque
ad \mpp{vespera, -ae (f):}{vesper}vesperam. 
\vnum{14}Quod cum vīdisset cognātus
eius omnia scīlicet quæ agēbat in populō, ait: ``Quid est hoc quod facis in
\mpp{plēbs, plēbis (f):}{populī multitudo}plēbe? Cūr sōlus sēdēs, et omnis populus
\mpp{præstōlārī:}{stāre ante aliquem et expectāre}præstōlātur dē māne usque ad vesperam?''

\vnum{15}Cui respondit
Moysēs: ``Venit ad mē populus quærēns sententiam Deī: 
\vnum{16}cumque acciderit
eīs aliqua \mpp{disceptātiō, -ōnis (f):}{e.g, unus dicit alterum malum fecisse, et alter se defendit -- hoc est disceptātiō}disceptātiō, veniunt ad mē ut
\mpp{iūdicium, -ī:}{quod iudex tradit}iūdicem inter eōs, et ostendam
\mpp{praeceptum, -ī:}{quod praecipitur}præcepta Deī, et lēgēs eius.''

\vnum{17}At ille: ``Nōn bonam, inquit, rem facis. 
\vnum{18}Stultō labōre cōnsūmeris et tū, et
populus iste quī tēcum est: ultrā vīrēs tuās est negōtium; sōlus illud
nōn poteris sustinēre. 
\vnum{19}Sed audī verba mea atque cōnsilia, et erit Deus
tēcum. Ēsto tū populō in hīs quæ ad Deum \mpp{pertinēre:}{e.g, colloquium
ad mē pertinet sī hominēs loquuntur dē mē}pertinent, ut referās quæ
dīcuntur ad eum: 
\vnum{20}ostendāsque populō \mpp{cæremōnia, -ae (f):}{res agendae ut sacra recte fiant}cæremōniās et
\mpp{rītus, -ūs (m):}{cæremōnia}rītum colendī,
viamque per quam \mpp{ingredī:}{intus īre}ingredī dēbeant, et opus quod
facere dēbeant. 
\vnum{21}\mpp{prōvidēre:}{curāre ut aliquid habeās}Prōvidē autem dē omnī plēbe virōs
\mpp{potēns, potentis (adj):}{qui multas res agere potest: e.g, dux, rēx}potentēs, et timentēs Deum, in quibus sit vēritās, et quī
ōderint \mpp{avāritia, -ae (f):}{amor pecuniae}avāritiam, et cōnstitue ex eīs
\mpp{tribūnus, -ī (m):}{dux multōrum}tribūnōs, et \mpp{centuriō, -ōnis (m):}{dux centum hominum}centuriōnēs, et
\mpp{quīnquāgēnāriō, -ōnis (m):}{dux quīnquāgīnta hominum}quīnquāgēnāriōs, et \mpp{decānus, -ī (m):}{dux decem hominum}decānōs, 
\vnum{22}quī
iūdicent populum omnī tempore: quidquid autem maius fuerit, referant ad
tē, et ipsī minōra \mpp{tantummodo =}{solum}tantummodo iūdicent: leviusque sit
tibi, \mpp{partior, partīrī, partitus:}{partēs facere, dīvidere}partītō in aliōs \mpp{onus, oneris (n):}{res quam difficile est
portare vel agere. e.g saccus lapidibus plenus, labor difficilis}onere.
\vnum{23}Sī hoc fēcerīs, implēbis imperium Deī, et præcepta eius poteris
\mpp{sustentāre:}{sustinēre}sustentāre: et omnis hic populus revertētur ad
loca sua cum pāce.'' 

\vnum{24}Quibus audītis, Moysēs fēcit omnia quæ ille
\mpp{suggerere:}{offerre consilium}suggesserat. 
\vnum{25}Et ēlēctīs virīs \mpp{strēnuus/a/um:}{impiger, industrius}strēnuīs
dē cūnctō Isrāēl, cōnstituit eōs prīncipēs populī, tribūnōs, et
centuriōnēs, et quīnquāgēnāriōs, et decānōs. 
\vnum{26}Quī iūdicābant
\mpp{plēbs, plēbis (f):}{populī multitudo}plēbem omnī tempore: quidquid autem gravius erat,
referēbant ad eum, faciliōra tantummodo iūdicantēs. 
\vnum{27}Dīmīsitque
cognātum suum: quī reversus abiit in terram suam. 
