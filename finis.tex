\invisiblechapter{Ad Fīnem}

\titleimg{ashburn2}

\mktitle{Ad Fīnem}
\thispagestyle{empty}

[Hīc incipit capitulum trīcēsimum quārtum:]

\vnum{1}Ac \mpp{deinceps:}{deinde}deinceps:
``\mpp{præcīdere:}{frangere in partes}Præcīde,'' ait, ``tibi duās tabulās \mpp{lapideus/a/um:}{ex
lapidibus factum}lapideās \mpp{īnstar + gen:}{similis}īnstar priōrum, et
scrībam super eās verba, quæ habuērunt tabulæ quās frēgistī. 
\vnum{2}Estō parātus
manē, ut ascendās statim in montem Sināī, stābisque mēcum super
\mpp{vertex, verticis (m):}{altissima pars}verticem montis. 
\vnum{3}Nūllus
ascendat tēcum, nec videātur \mpp{quispiam =}{aliquis}quispiam per tōtum montem:
bovēs quoque et ovēs nōn pāscantur ē contrā.''

\vnum{4}\mpp{excidere:}{secare partem alicuis et capere illam partem}Excidit ergō
duās tabulās lapideās, quālēs anteā fuerant: et dē nocte
\mpp{cōnsurgere:}{surgere}cōnsurgēns ascendit in montem Sināī, sīcut
\mpp{praecipere:}{imperāre}præcēperat eī Dominus, portāns sēcum tabulās. 
\vnum{5}Cumque dēscendisset Dominus per nūbem, stetit Moysēs cum
eō, invocāns nōmen Dominī. 

\vnum{6}Quō trānseunte cōram eō, ait: \mpp{dominātor, -ōris (m):}{qui dominus est multārum rērum}``Dominātor
Domine Deus, \mpp{misericors, -cordis (adj):}{qui facile dolorem sentit
propter aliquem miserum; clemens}misericors et clēmēns, patiēns et multæ
\mpp{miserātiō, -ōnis (f) <}{miser}miserātiōnis, ac \mpp{vērāx, vērācis
(adj):}{qui vera dicit}vērāx, 
\vnum{7}quī cūstōdīs
\mpp{misericordia, -ae (f):}{quod sentimus dum vidēmus aliquem
patī}misericordiam in mīllia; quī aufers \mpp{inīquitās, -ātis (f):}{rēs
mala ab homine facta; mala anima propter rem malam factam}inīquitātem, et
scelera, atque \mpp{peccātum, -ī (n):}{res contra legem acta}peccāta, nūllusque
apud tē per sē \mpp{innocēns, -centis (adj):}{qui non nocet}innocēns est; quī reddis inīquitātem patrum
fīliīs, ac \mpp{nepos, nepōtis (m):}{filius filiī vel filia filiae}nepōtibus in tertiam et quārtam
\mpp{prōgeniēs, -ēī (f):}{filiī vel filiae alicuius, et filiī vel filiae
illōrum, et filiī vel filiae illōrum, etc.}prōgeniem.''

\vnum{8}\mpp{festīnus/a/um:}{celeriter aliquid
agens}Festīnusque Moysēs, \mpp{curvātus/a/um:}{quod non habet formam rectam}curvātus est \mpp{prōnus/a/um:}{quod flectitur ad aliquid ante sē}prōnus in terram, et
adōrāns 
\vnum{9}ait: ``Sī invēnī grātiam in cōnspectū tuō, Domine,
\mpp{obsecrāre:}{orāre, precārī}obsecrō ut gradiāris nōbīscum (populus enim
dūræ \mpp{cervīx, cervīcis (f):}{posterior pars collī; collum}cervīcis est)
et auferās inīquitātēs nostrās atque peccāta, nōsque possideās.''

\vnum{10}Respondit Dominus: ``Ego \mpp{inīre <}{in + īre}inībō \mpp{pactum, -ī (n):}{quod inter aliquōs
convenit}pactum videntibus cūnctīs: signa faciam quæ numquam vīsa sunt
super terram, nec in ūllīs gentibus, ut cernat populus iste, in cuius es
mediō, opus Dominī terribile quod factūrus sum. 
\vnum{11}\mpp{observāre:}{colere, servare}Observā cūncta quæ hodiē \mpp{mandāre:}{tradere}mandō tibi...''

[Deus plūrēs lēgēs Moȳsī dat.] 

\vnum{27}Dīxitque Dominus ad
Moysēn: ``Scrībe tibi verba hæc, quibus et tēcum et cum
Isrāēl \mpp{pangō, pangere, pepigisse, pactum:}{constituere, statuere}pepigī fœdus.''

\vnum{28}Fuit ergō ibi cum
Dominō quadrāgintā diēs et quadrāgintā noctēs: pānem nōn comēdit, et aquam
nōn bibit, et scrīpsit in tabulīs verba fœderis decem. 
\vnum{29}Cumque
dēscenderet Moysēs dē monte Sināī, tenēbat duās tabulās \mpp{testimōnium,
-ī (n):}{id quod ā teste dīcitur}testimōniī, et ignōrābat quod
\mpp{cornūtus/a/um:}{cornua habens}cornūta esset faciēs sua ex
\mpp{cōnsortiō, -ōnis (f):}{congregatiō}cōnsortiō
sermōnis Dominī. 
\vnum{30}Videntēs autem Aarōn et fīliī Isrāēl
cornūtam Moȳsī faciem, timuērunt prope accēdere. 
\vnum{31}Vocātīque ab eō, reversī sunt tam Aarōn, quam prīncipēs
\mpp{synagōga, -ae (f):}{grex hominum}synagōgæ. Et postquam locūtus est ad eōs, 
\vnum{32}vēnērunt ad
eum etiam omnēs fīliī Isrāēl: quibus præcēpit cūncta quæ audierat ā Dominō
in monte Sināī. 
\vnum{33}Implētīsque sermōnibus, posuit \mpp{vēlāmen, vēlaminis (n):}{vestis quae faciem operit}vēlāmen
super faciem suam. 
\vnum{34}Quod \mpp{ingredī:}{intus īre}ingressus ad Dominum,
et loquēns cum eō, auferēbat dōnec exīret, et tunc loquēbātur ad fīliōs
Isrāēl omnia quæ sibi fuerant imperāta. 
\vnum{35}Quī vidēbant faciem ēgredientis
Moȳsī esse cornūtam, sed operiēbat ille rūrsus faciem suam,
\mpp{sīquandō:}{sī aliquando}sīquandō loquēbātur ad eōs.

[Moysēs lēgēs Deī populō dedit,
deinde perrēxit populus \mimg{tab2}{tabernaculum, -ī (n)}tabernāculum Deī cōnficere.
Tabernāculō perfectō, nūbēs (Deus) tabernāculum operuit.]

[Capitulum Quadrāgēsimum:]

\vnum{34}Sīquandō nūbēs tabernāculum dēserēbat,
proficīscēbantur fīliī Isrāēl per \mpp{turma, -ae (f):}{multitūdō}turmās
suās: 
\vnum{35}Sī pendēbat \mpp{dēsuper:}{ē locō altiore}dēsuper, manēbant in
eōdem locō. 
\vnum{36}Nūbēs \mpp{quippe =}{certe}quippe Dominī
\mpp{incubāre:}{super aliquem rem cubāre, requiescere}incubābat per diem tabernāculō, et ignis in nocte,
videntibus cūnctīs populīs Isrāēl per cūnctās \mpp{mānsiō, -ōnis
(f):}{locus ubi iter facientes noctu manent}mānsiōnēs suās.

[Fīnis Librī Exodī] 
