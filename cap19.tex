\chapter{}

\titleimg{tencommand}

\mktitle{Capitulum Undevicesimum}
\thispagestyle{empty}

\cstart{M}{ēnse} tertiō \mpp{ēgressiō, -ōnis (f) <}{ēgredī}ēgressiōnis Isrāēl dē
terrā Ægyptī, in diē hāc vēnērunt in \mpp{sōlitūdo, -inis ($<$
solus/a/um):}{deserta; locus ubi nemo vel unus habitat}sōlitūdinem Sināī. 
\vnum{2}Nam profectī dē Raphidim, et pervenientēs usque in dēsertum Sināī,
\mpp{castramētārī:}{castra ponere}castramētātī sunt in eōdem locō, ibique
Isrāēl fīxit \mimg{tab2}{tentōrium, -ī (n)}tentōria ē regiōne montis. 

\vnum{3}Moysēs autem ascendit ad Deum: vocāvitque eum Dominus dē
monte, et ait: ``Hæc dīcēs domuī Iācōb, et
\mpp{annūntiāre:}{nuntiāre}annūntiābis fīliīs Isrāēl: 
\vnum{4}Vōs ipsī vīdistis quæ fēcerim
Ægyptiīs, quōmodo portāverim vōs super ālās aquilārum, et assūmpserim mihi.
\vnum{5}Sī ergō audierītis vōcem meam, et cūstōdieritis \mpp{pactum, -ī
(n):}{quod inter aliquōs convenit}pactum meum, eritis mihi in pecūlium dē
cūnctīs populīs: mea est enim omnis terra: 
\vnum{6}et vōs eritis mihi in
\mpp{rēgum, rēgī (n):}{quod rēx rēgit}rēgnum \mpp{sacerdōtālis, -e (adj) <}{sacerdōs}sacerdōtāle, et gēns
\mpp{sanctus/a/um:}{deo pertinens}sāncta. Hæc sunt verba quæ loqueris ad fīliōs
Isrāēl.''

\vnum{7}Venit Moysēs: Et convocātīs maiōribus nātū populī, exposuit
omnēs sermōnēs quōs \mpp{mandātum, -ī (n):}{id quod imperatum vel traditum
est}mandāverat Dominus. 

\vnum{8}Responditque omnis populus simul: ``Cūn\-cta quæ
locūtus est Dominus, faciēmus.''

Cumque retulisset Moysēs verba populī ad
Dominum, \vnum{9}ait eī Dominus: ``Iam nunc veniam ad tē in
\mpp{cālīgō, cālīginis (f):}{aer per quem difficile est vidēre}cālīgine nūbis, ut audiat mē populus loquentem ad tē, et
crēdat tibi in perpetuum.''\mimg{bucina}{buccina, -ae (f)}


Nūntiāvit ergō Moysēs verba populī ad Dominum.  \vnum{10}Quī dīxit eī: ``Vāde ad populum, et \mpp{sānctificāre:}{sanctum
facere}sānctificā illōs hodiē, et crās, laventque vestīmenta sua. 
\vnum{11}Et sint parātī in diem tertium: in diē enim tertiā dēscendet Dominus cōram
omnī \mpp{plēbs, plēbis (f):}{populus; praecipue hominēs quī pecuniosī non sunt ac minimam potestatem habent}plēbe super montem Sināī. 
\vnum{12}Cōnstituēs\-que \mpp{terminus, -ī (m):}{finis terrae}terminōs populō per \mpp{circuitus, -ūs
(m):}{locus circum alterum locum}\mimg{rock}{lapis, lapidis (m)}circuitum, et dīcēs ad eōs: Cavēte nē
ascendātis in montem, nec tangātis fīnēs illīus: omnis quī tetigerit
montem, morte moriētur.
\vnum{13}Manus nōn tanget eum, sed
lapidibus \mpp{opprimere:}{premere aliquem ne surgere possit}opprimētur, aut \mpp{cōnfodere:}{vulnerāre}cōnfodiētur \mpp{iaculum, -ī (n):}{quiquid iacī potest ut aliquis vulnerētur}iaculīs:
sīve \mpp{iūmentum, -ī (n):}{animal utile ad gerendum vel trahendum e.g,
equī, bovēs, et cetera}iūmentum fuerit, sīve homō, nōn vīvet: cum cœperit
\mpp{clangere:}{facere sonum magnum ut tuba}clangere buccina, tunc ascendant in montem.''

\vnum{14}Dēscenditque Moysēs dē monte ad populum, et sānctificāvit eum. Cumque
lāvissent vestīmenta sua, 
\vnum{15}ait ad eōs: ``Ēstote parātī in diem tertium, et
nē appropinquētis uxōribus vestrīs.''

\vnum{16}Iamque advēnerat tertius diēs, et
manē inclāruerat: et ecce cœpērunt audīrī \mpp{tonitruum, -ī (n):}{tonitrus}tonitrua, ac
\mpp{micāre:}{lucem mittere ut ignis, stellae, et cetera}micāre fulgura, et nūbēs \mpp{densus/a/um:}{cuius partēs
inter sē premunt}dēnsissima operīre montem, \mpp{clangor, -ōris (m):}{strepitus}clangorque
buccinæ \mpp{vehemens, -entis (adj):}{ferox, iratus, severus}vehementius
\mpp{perstrepere:}{strepitum magnum facere}perstrepēbat: et timuit populus quī erat in castrīs. 

\begin{figure}[hp]
    \begin{minipage}[hbp]{0.5\linewidth}
        \centering
        \includegraphics{root}
        \caption{rādīx, rādīcis (f)}
    \end{minipage}%
    \begin{minipage}[hbp]{0.5\linewidth}
        \centering
        \includegraphics{fumus}
        \caption{fūmus, -ī (m)}
    \end{minipage}
\end{figure}

\vnum{17}Cumque ēdūxisset
eōs Moysēs in \mpp{occursus, -ūs (m) <}{occurere}occursum Deī dē locō castrōrum, stetērunt ad
\mpp{rādīx montis $\leftrightarrow$}{altissima pars montis}rādīcēs montis. 
\vnum{18}Tōtus autem mōns Sināī
\mpp{fūmāre:}{fumum mittere}fūmābat, eō quod dēscendisset Dominus super eum in igne:
et ascenderet fūmus ex eō quasi dē \mimg{furnus}{fornax, fornācis (f)}fornāce,
eratque omnis mōns terribilis. 
\vnum{19}Et \mpp{sonitus, -ūs (m):}{sonus magnus, strepitus}sonitus buccinæ
\mpp{paulātim:}{tardē plus et plus et plus vel magis et magis et magis}paulātim crēscēbat in maius, et \mpp{prōlixus/a/um:}{longus}prōlixius
tendēbātur: Moysēs loquēbātur, et Deus respondēbat eī. 
\vnum{20}Dēscenditque
Dominus super montem Sināī in ipsō montis \mpp{vertex, verticis
(m):}{altissima pars}vertice, et vocāvit Moysen in
\mpp{cacūmen, cacūminis (m):}{altissima pars}cacūmen eius.

Quō cum ascendisset, 
\vnum{21}dīxit ad eum:
``Dēscende, et \mpp{contestāre populum:}{fac tē testem verbōrum meōrum populō; narra populō ea quae tibi dixī}contestāre populum: nē forte velit
\mpp{trānscendere:}{transīre}trānscendere terminōs ad videndum Dominum, et
pereat ex eīs plūrima multitūdō. 
\vnum{22}Sacerdōtēs quoque quī accēdunt ad
Dominum, sānctificentur, nē percutiat eōs.''

\vnum{23}Dīxitque Moysēs ad Dominum:
``Nōn poterit \mpp{vulgus, -ī (n):}{multitudo, turba}vulgus ascendere in
montem Sināī: tū enim \mpp{testificārī:}{testis esse}testificātus es, et iussistī, dīcēns:
Pōne \mpp{terminus, -ī (m):}{res quae indicat finis reī, e.g: lapis positus inter agrōs}terminōs circā montem, et sānctificā illum.''

\vnum{24}Cui ait Dominus:
``Vāde, dēscende: as\-cendēsque tū, et Aarōn tēcum: sacerdōtēs autem et
populus nē trānseant terminōs, nec ascendant ad Dominum, nē forte
interficiat illōs.''

\vnum{25}Dēscenditque Moysēs ad populum, et omnia nārrāvit
eīs.
