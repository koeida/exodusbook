\chapter{}

\titleimg{arms.jpg}

\mktitle{Capitulum Undevicesimus}
\thispagestyle{empty}

\vnum{1}Mēnse tertiō \mpp{ēgressiō, -ōnis (f) <}{ēgredī}ēgressiōnis Isrāēl dē
terrā Ægyptī, in diē hāc vēnērunt in \mpp{sōlitūdo, -inis ($<$
solus/a/um):}{deserta; locus ubi nemo vel unus habitat}sōlitūdinem Sināī. 
\vnum{2} Nam profectī dē Raphidim, et pervenientēs usque in dēsertum Sināī,
\mpp{castramētārī:}{castra ponere}castramētātī sunt in eōdem locō, ibique
Isrāēl fīxit \mimg{tab2}{tentōrium, -ī (n)}tentōria ē regiōne montis. 

\vnum{3}Moysēs autem ascendit ad Deum: vocāvitque eum Dominus dē
monte, et ait: ``Hæc dīcēs domuī Iācōb, et
\mpp{annūntiāre:}{nuntiāre}annūntiābis fīliīs Isrāēl: 
\vnum{4}Vōs ipsī vīdistis quæ fēcerim
Ægyptiīs, quōmodo portāverim vōs super ālās aquilārum, et assūmpserim mihi.
\vnum{5}Sī ergō audierītis vōcem meam, et cūstōdieritis \mpp{pactum, -ī
(n):}{quod inter aliquōs convenit}pactum meum, eritis mihi in pecūlium dē
cūnctīs populīs: mea est enim omnis terra: 
\vnum{6}et vōs eritis mihi in
\mpp{????}{????}rēgnum \mpp{????}{????}sacerdōtāle, et gēns
\mpp{sanctus:}{deo pertinens}sānctā. Hæc sunt verba quæ loqueris ad fīliōs
Isrāēl.''

\vnum{7}Venit Moysēs: et convocātīs maiōribus nātū populī, exposuit
omnēs sermōnēs quōs \mpp{mandātum, -ī (n):}{id quod imperatum vel traditum
est}mandāverat Dominus. 

\vnum{8}Responditque omnis populus simul: ``Cūncta quæ
locūtus est Dominus, faciēmus.''

Cumque retulisset Moysēs verba populī ad
Dominum, \vnum{9}ait eī Dominus: ``Iam nunc veniam ad tē in
\mpp{????}{????}cālīgine nūbis, ut audiat mē populus loquentem ad tē, et
crēdat tibi in perpetuum. Nūntiāvit ergō Moysēs verba populī ad Dominum.''

\vnum{10}Quī dīxit eī: ``Vāde ad populum, et \mpp{sānctificāre:}{sanctum
facere}sānctificā illōs hodiē, et crās, laventque vestīmenta sua. 
\vnum{11}Et sint parātī in diem tertium: in diē enim tertia dēscendet Dominus cōram
omnī \mpp{????}{????}plēbe super montem Sināī. 
\vnum{12}Cōnstituēsque \mpp{terminus, -ī:}{finis terrae}terminōs populō per \mpp{circuitus, -ūs
(m):}{locus circum alterum locum}circuitum, et dīcēs ad eōs: Cavēte nē
ascendātis in montem, nec tangātis fīnēs illīus: omnis quī tetigerit
montem, morte moriētur. 
\vnum{13}Manus nōn tanget eum, sed
\mimg{rock}{lapis, lapidis (m)}lapidibus \mpp{opprimere:}{efficere ut aliquis surgere nōn
possit}opprimētur, aut \mpp{????}{????}cōnfodiētur \mpp{????}{????}iaculīs
: sīve \mpp{iūmentum, -ī (n):}{animal utile ad gerendum vel trahendum e.g,
equī, bovēs, et cetera}iūmentum fuerit, sīve homō, nōn vīvet: cum cœperit
\mpp{????}{????}clangere \mpp{????}{????}buccina, tunc ascendant in montem.''

\vnum{14}Dēscenditque Moysēs dē monte ad populum, et sānctificāvit eum. Cumque
lāvissent vestīmenta sua, 
\vnum{15}ait ad eōs: ``Ēstote parātī in diem tertium, et
nē appropinquētis uxōribus vestrīs.''

\vnum{16}Iamque advēnerat tertius diēs, et
manē inclāruerat: et ecce cœpērunt audīrī \mpp{????}{????}tonitrua, ac
\mpp{????}{????}micāre fulgura, et nūbēs \mpp{densus/a/um:}{cuius partēs
inter sē premunt}dēnsissima operīre montem, \mpp{????}{????}clangorque
\mpp{????}{????}buccinæ \mpp{vehemens:}{ferox, iratus, severus}vehementius
perstrepēbat: et timuit populus quī erat in castrīs. 
\vnum{17}Cumque ēdūxisset
eōs Moysēs in occursum Deī dē locō castrōrum, stetērunt ad
\mpp{????}{????}rādīcēs montis. 
\vnum{18}Tōtus autem mōns Sināī
\mpp{????}{????}fūmābat, eō quod dēscendisset Dominus super eum in igne:
et ascenderet \mpp{????}{????}fūmus ex eō quasi dē \mpp{????}{????}fornāce,
eratque omnis mōns terribilis. 
\vnum{19}Et \mpp{????}{????}sonitus buccinæ
\mpp{????}{????}paulātim crēscēbat in maius, et \mpp{????}{????}prōlixius
tendēbātur: Moysēs loquēbātur, et Deus respondēbat eī. 
\vnum{20}Dēscenditque
Dominus super montem Sināī in ipsō montis \mpp{vertex, verticis
(m):}{altissima pars}vertice, et vocāvit \mpp{????}{????}Moysen in
\mpp{????}{????}cacūmen eius. Quō cum ascendisset, 
\vnum{21}dīxit ad eum:
Dēscende, et \mpp{????}{????}contestāre populum: nē forte velit
\mpp{trānscendere:}{transīre}trānscendere terminōs ad videndum Dominum, et
pereat ex eīs plūrima multitūdō. 
\vnum{22}Sacerdōtēs quoque quī accēdunt ad
Dominum, sānctificentur, nē percutiat eōs. 
\vnum{23}Dīxitque Moysēs ad Dominum:
Nōn poterit \mpp{vulgus, -ī (n):}{multitudo, turba}vulgus ascendere in
montem Sināī: tū enim \mpp{????}{????}testificātus es, et iussistī, dīcēns
: Pōne terminōs circā montem, et sānctificā illum. 
\vnum{24}Cui ait Dominus:
Vāde, dēscende: ascendēsque tū, et Aarōn tēcum: sacerdōtēs autem et
populus nē trānseant terminōs, nec ascendant ad Dominum, nē forte
interficiat illōs. 
\vnum{25}Dēscenditque Moysēs ad populum, et omnia nārrāvit
eīs.
